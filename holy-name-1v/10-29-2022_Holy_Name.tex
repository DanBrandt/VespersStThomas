\documentclass{../vespers-booklet}
\usepackage{multicol}

\begin{document}

\chapter*{First Vespers of the Feast of the Holy Name of Jesus}

\begin{rubricbox}

{\color{red}When the Officiant kneels, all \textbf{kneel} and pray silently.
Then, when the Officiant stands, all \textbf{stand} and say silently one \textit{Pater noster} (Our Father) and \textit{Ave Maria} (Hail Mary).
Then all make the sign of the cross with the Officiant as he intones:}

\end{rubricbox}

 \gresetinitiallines{1}
\gregorioscore{../common/deus-in-adjutorium-solemn}

\textit{
O God, come to my assistance.
\Vbar.~O Lord, make haste to help me.
Glory be to the Father, and to the Son, and to the Holy Spirit,
as it was in the beginning, is now, and ever shall be, world without end. Amen.
Praise to Thee, O Lord, King of endless glory.}

\vfill\pagebreak

% FIRST PSALM: PSALM 109

\section*{Psalm 109}

\textit{\textnormal{Ant. 1.} He shall be called * Peaceable, and his throne shall be most firm for ever.
 \textnormal{Ps.} The Lord said to my Lord: * Sit thou at my right hand:}
 
 \begin{rubricbox}

{\color{red}All remain standing throughout the first antiphon.
After the psalm is intoned by the Cantor, all \textbf{sit} at the asterisk.}

\end{rubricbox}

\gresetinitiallines{1}
\gregorioscore{ps109-antiphon}

\gresetinitiallines{0}
\gregorioscore{ps109-intonation}

 \begin{latinenglishsection}

\latinenglish{

	\input{../psalms/ps109-8} %%

}{
	\input{../psalms/english/ps109} %%
}

\end{latinenglishsection}

\begin{rubricbox}

{\color{red}The antiphon is repeated: \textit{Pacíficus\dots}} %%

\end{rubricbox}

% SECOND PSALM: PSALM 110

\section*{Psalm 110}

\textit{\textnormal{Ant. 2.} His kingdom * is an everlasting kingdom, and all kings shall serve him, and shall obey him.
 \textnormal{Ps.} I will praise thee, O Lord, with my whole heart; * in the council of the just, and in the congregation.}
 
 \begin{rubricbox}

{\color{red}All remain standing throughout the first antiphon.
After the psalm is intoned by the Cantor, all \textbf{sit} at the asterisk.}

\end{rubricbox}

\gresetinitiallines{1}
\gregorioscore{ps110-antiphon}

\gresetinitiallines{0}
\gregorioscore{ps110-intonation}

 \begin{latinenglishsection}

\latinenglish{

	\input{../psalms/ps110-8} %%

}{
	\input{../psalms/english/ps110} %%
}

\end{latinenglishsection}

\begin{rubricbox}

{\color{red}The antiphon is repeated: \textit{Regnum ejus\dots}} %%

\end{rubricbox}

% THIRD PSALM: PSALM 111

\section*{Psalm 111}

\textit{\textnormal{Ant. 3.} Behold a man, the Orient * is his name: he shall sit and rule, and shall speak peace to the Gentiles.
 \textnormal{Ps.} Blessed is the man that feareth the Lord: * he shall delight exceedingly in his commandments.}
 
 \begin{rubricbox}

{\color{red}All remain standing throughout the first antiphon.
After the psalm is intoned by the Cantor, all \textbf{sit} at the asterisk.}

\end{rubricbox}

\gresetinitiallines{1}
\gregorioscore{ps111-antiphon}

\gresetinitiallines{0}
\gregorioscore{ps111-intonation}

 \begin{latinenglishsection}

\latinenglish{

	\input{../psalms/ps111-7} %%

}{
	\input{../psalms/english/ps111} %%
}

\end{latinenglishsection}

\begin{rubricbox}

{\color{red}The antiphon is repeated: \textit{Ecce Vir Oriens\dots}} %%

\end{rubricbox}

% FOURTH PSALM: PSALM 112

\section*{Psalm 112}

\textit{\textnormal{Ant. 4.} The Lord * is our judge, the Lord is our lawgiver: the Lord is our king, he will save us.
 \textnormal{Ps.} Praise the Lord, ye children: * praise ye the name of the Lord.}
 
 \begin{rubricbox}

{\color{red}All remain standing throughout the first antiphon.
After the psalm is intoned by the Cantor, all \textbf{sit} at the asterisk.}

\end{rubricbox}

\gresetinitiallines{1}
\gregorioscore{ps112-antiphon}

\gresetinitiallines{0}
\gregorioscore{ps112-intonation}

 \begin{latinenglishsection}

\latinenglish{

	\input{../psalms/ps112-3} %%

}{
	\input{../psalms/english/ps112} %%
}

\end{latinenglishsection}

\begin{rubricbox}

{\color{red}The antiphon is repeated: \textit{Dóminus\dots}} %%

\end{rubricbox}

\vfill\pagebreak

% FIFTH PSALM: PSALM 116

\section*{Psalm 116}

\textit{\textnormal{Ant. 5.} Behold, I have given thee * to be the light of the Gentiles, that thou mayst be my salvation even to the farthest part of the earth.
 \textnormal{Ps.} Praise the Lord, all ye nations: * praise him, all ye people.}
 
 \begin{rubricbox}

{\color{red}All remain standing throughout the first antiphon.
After the psalm is intoned by the Cantor, all \textbf{sit} at the asterisk.}

\end{rubricbox}

\gresetinitiallines{1}
\gregorioscore{ps116-antiphon}

\gresetinitiallines{0}
\gregorioscore{ps116-intonation}

 \begin{latinenglishsection}

\latinenglish{

	2. Quóniam confirmáta est super nos misericórdia \textbf{e}jus:~* {\color{red}\textit{(stand)}}
	et véritas Dómini manet \textit{in} \textit{æ}\textbf{tér}num.

{\color{red}\textit{(bow)}} Glória Patri, et \textbf{Fí}lio,~*
	et Spirí\textit{tu}\textit{i} \textbf{Sanc}to.

{\color{red}\textit{(rise)}} Sicut erat in princípio, et nunc, et \textbf{sem}per,~*
	et in s\'{\ae}cula sæcu\textit{ló}\textit{rum}. \textbf{A}men. %%

}{
	\input{../psalms/english/ps116} %%
}

\end{latinenglishsection}

\begin{rubricbox}

{\color{red}The antiphon is repeated: \textit{ Ecce dedi te\dots}} %%

\end{rubricbox}

\vfill\pagebreak

%TODO: Verify that the little chapter is fitting for the feast

\section*{Little Chapter (Col 1:12-13)}

\textit{\color{red}The Officiant leads the Little Chapter:}

\begin{latinenglishsection}

\latinenglish{
	Fratres: Grátias ágimus Deo Patri, qui dignos nos fecit in partem sortis sanctórum in lúmine,\GreDagger qui erípuit nos de potestáte \textit{tene}\textbf{brá}rum, et tránstulit in regnum Fílii dilectiónis suæ.
	\Rbar.~Deo grátias.
}{
	Brethren: We give thanks to God the Father, who hath made us worthy to be partakers of the lot of the saints in light, who hath delivered us from the power of darkness, and hath translated us into the kingdom of the Son of his love.
	 \Rbar.~Thanks be to God.
}

\end{latinenglishsection}

% TODO: Verify that the hymn is correct for the feast (including the responsory after the hymn)

\section*{Hymn}

\textit{\color{red}The Cantor leads the hymn:}

\gresetinitiallines{1}
\gregorioscore{../hymns/te-saeculorum-principem}

{\itshape

	1. To thee, O Prince of all that be,
	Thou Christ, O King eternally;
	O Framer of the mind and heart,
	Our one true Judge we say thou art.
	
	2. The wicked protest, wail and cry,
	Christ Jesus’ reign they would deny;
	Rejoice we at thy glorious name,
	Thou Highest King we do proclaim.
	
	3. O Christ! The Source of all our peace,
	Make all our sinful thoughts to cease;
	And still in us our loves misplaced,
	As Thy one sheepfold be we embraced.
	
	4. For this, hanging on cruel tree,
	With arms outstretched, for all to see;
	His heart is pierced by soldier’s spear,
	Revealing burning love most dear.
	
	5. From this the altar of the tree
	Thy blood flows forth from Calvary;
	As wine to us it doth appear,
	To thine own heart it draws us near.
	
	6. Thou Governor of all that be,
	May all thy creatures honour thee;
	All those who rule, O Lord renew!
	Source of all precepts just and true.
	
	7. To regal glory, all submit,
	All crowns and honours we do remit;—
	To thy scepter—so sweet and mild!
	Submit we as a little child.
	
	8. All glory be, Jesu, to thee,
	Thy scepter over all that be;
	All glory, as is ever meet,
	To Father and to Paraclete.
	Amen.
}

\textit{\color{red}The Cantor says the following before all reply afterwards:}

\gresetinitiallines{0}
\gabcsnippet{
(c3) <sp>V/</sp>. Da(h)ta(h) est(h) mi(h)hi(h) om(h)nis(h) po(h)tés(h)tas.(g'_/hvGF'E/fgf.) (::)
}

\gresetinitiallines{0}
\gabcsnippet{
(c3) <sp>R/</sp>. In(h) cæ(h)lo(h) et(h) in(h) ter(h)ra.(g'_/hvGF'E/fgf.) (::)
}

\textit{\Vbar.~Given to me is all power.
\Rbar.~In heaven and in earth.}

\vfill\pagebreak

\section*{Magnificat}

\textit{\textnormal{Ant Magn.} The Lord God * shall give unto him the throne of David, his father: and he shall reign in the house of Jacob for ever, and of his kingdom there shall be no end, alleluia.
\textnormal{Cant.} My soul doth magnify the Lord: and my spirit hath rejoiced in God my Saviour.}

\begin{rubricbox}

{\color{red}The Cantor leads by intoning the antiphon and the first verse.}

\end{rubricbox}

\gresetinitiallines{1}
\gregorioscore{magnificat-antiphon-only}

\begin{rubricbox}

{\color{red}All \textbf{stand} and make the sign of the cross with the Cantor.}

\end{rubricbox}

\gresetinitiallines{0}
\gregorioscore{magnificat-intonation}

 \begin{latinenglishsection}

\latinenglish{	
3. Quia respéxit humilitátem ancíllæ \textbf{su}æ:~*
	ecce enim ex hoc beátam me dicent omnes gene\textit{ra}\textit{ti}\textbf{ó}nes.

4. Quia fecit mihi magna qui \textbf{pot}ens est:~* 
	et sanctum \textit{no}\textit{men} \textbf{e}jus.

5. Et misericórdia ejus a progénie in pro\textbf{gé}nies~*
	timén\textit{ti}\textit{bus} \textbf{e}um.

6. Fecit poténtiam in bráchio \textbf{su}o:~*
	dispérsit supérbos mente \textit{cor}\textit{dis} \textbf{su}i.

7. Depósuit poténtes de \textbf{se}de,~*
	et exal\textit{tá}\textit{vit} \textbf{hú}miles.

8. Esuriéntes implévit \textbf{bo}nis:~*
	et dívites dimí\textit{sit} \textit{in}\textbf{á}nes.

9. Suscépit Israël púerum \textbf{su}um,~*
	recordátus misericór\textit{di}\textit{æ} \textbf{su}æ.

10. Sicut locútus est ad patres \textbf{nos}tros,~*
	Abraham et sémini e\textit{jus} \textit{in} \textbf{s\'{\ae}}cula.

}{	
	\input{../psalms/english/magnificat}
}

\end{latinenglishsection}

\textit{\color{red}(bow)}Glória Patri, et \textbf{Fí}lio,~* 
	et Spirí\textit{tu}\textit{i} \textbf{Sanc}to.

\textit{\color{red}(rise)} Sicut erat in princípio, et nunc, et \textbf{sem}per,~*
	et in s\'{\ae}cula sæcu\textit{ló}\textit{rum}. \textbf{A}men.


\begin{rubricbox}

{\color{red}All \textbf{sit} and repeat the antiphon: \textit{Dabit illi\dots}, %%
then \textbf{stand} for the prayer.}

\end{rubricbox}

%TODO: Verify (with the antiphonary) that the collect is proper for the season. If it is not in antiphonary, use the missal for the feast.

\section*{Collect}

\textit{\color{red}The Officiant leads the collect:}

\begin{latinenglishsection}

\latinenglish{
	\Vbar.~Dómine exáudi oratiónem meam.\\
	\Rbar.~Et clamor meus ad te véniat.
	
	Orémus.
	Omnípotens sempitérne Deus, qui in dilécto Fílio tuo,\\ universórum Rege, ómnia instauráre voluísti: concéde propítius; ut cunctæ famíliæ géntium, peccáti vúlnere disgregátæ, ejus suavíssimo subdántur império:
	Qui tecum vivit et regnat in unitáte Spíritus Sancti, Deus, per ómnia s\'{\ae}cula sæculórum.
	\Rbar.~Amen.
}{
	Lord, hear my prayer.
	\Rbar.~And let my cry come unto Thee.
	
	Let us pray.
	Almighty and everlasting God, who in thy beloved Son, the King of the whole world, hast willed to restore all things: mercifully grant that all the families of nations, now kept apart by the wound of sin, may be brought under the sweet yoke of his rule.
	Who with thee liveth and reigneth, in the unity of the Holy Spirit, God, world without end.
	\Rbar.~Amen.
}
\end{latinenglishsection}

\vfill\pagebreak

%TODO: Add commemorations for the date

\textit{\color{red}For commemorations, the Cantor intones the antiphon and says the responsorial prayer afterwards. The Officiant prays the associated collect.}

\section*{Commemoration of the Twenty-first Sunday after Pentecost}

\textit{\textnormal{Ant.} I saw the Lord sitting upon a throne high and lifted up, and the whole earth was full of His glory, and His train filled the Temple.}

\gresetinitiallines{1}
\gregorioscore{../commemorations/an--vidi_dominum_sedentem_--solesmes_1934}

\begin{latinenglishsection}

\latinenglish{
	\Vbar.~Vespertína orátio ascéndat ad te, Dómine.\\
	\Rbar.~Et descéndat super nos\\ misericórdia tua.
	
	Orémus.
	Famíliam tuam, qu\'{\ae}sumus, Dómine, contínua pietáte custódi: ut a cunctis adversitátibus te\\ protegénte, sit líbera; et in bonis áctibus tuo nómini sit devóta.
	Per Dóminum nostrum Jesum Christum, Fílium tuum: qui tecum vivit et regnat in unitáte Spíritus Sancti, Deus, per ómnia s\'{\ae}cula sæculórum.
}{
	Thou hast crowned him with glory and honor, O Lord.
	\Rbar.~And hast set him over the works of Thy hands.
	
	Let us pray.
	O Lord, we beseech thee to keep thine household in continual godliness, that, through thy protection, it may be free from all adversities, and devoutly given to serve thee in good works, to the glory of thy Name.
	Through Jesus Christ, thy Son our Lord, Who liveth and reigneth with thee, in the unity of the Holy Ghost, God, world without end.
}

\end{latinenglishsection}

\textit{\color{red}The Officiant leads the following:}

\begin{latinenglishsection}

\latinenglish{
	\Vbar.~Dómine exáudi oratiónem meam.\\
	\Rbar.~Et clamor meus ad te véniat.
}{
	\Vbar.~Lord, hear my prayer. \Rbar.~And let my cry come unto Thee.
	\Vbar.~Let us bless the Lord. \Rbar.~Thanks be to God.
}

\end{latinenglishsection}

\vfill\pagebreak

\textit{\color{red}The Cantor leads the Benedicamus:}

\gresetinitiallines{1}
\gregorioscore{../common/benedicamus-1v-double}

\textit{\color{red}The Officiant leads the following:}

\begin{latinenglishsection}

\latinenglish{
	\Vbar. Fidélium ánimæ, per\\ misericórdiam Dei, requiéscant in pace. \\
	\Rbar. Amen.
}{
	May the souls of the faithful departed, through the mercy of God, rest in peace. \Rbar.~Amen.
}

\latinenglish{
	Pater noster \textit{(silently)}.
}{
	Our Father\dots
}

\latinenglish{
	\Vbar. Dóminus det nobis suam pacem. \\
	\Rbar. Et vitam ætérnam. Amen.
}{
	May the Lord grant us his peace. \Rbar.~And life eternal. Amen.
}

\end{latinenglishsection}

%TODO: Add the Marian Anthem for the season and verify that the oration afterwards is correct

\vfill\pagebreak

\section*{Marian Anthem}

\textit{\color{red}The Cantor leads the Marian anthem and responses afterwards; the Officiant leads the ending collect:}

\gresetinitiallines{1}
\gregorioscore{../marian-anthems/salve-regina-solemn-tone}

{\itshape
	Hail, holy Queen, Mother of mercy, our life, our sweetness and our hope. 
	To thee do we cry, poor banished children of Eve. 
	To thee do we send up our sighs, mourning and weeping in this valley of tears. 
	Turn, then, most gracious advocate, thine eyes of mercy toward us, 
	and after this, our exile, show unto us the blessed fruit of thy womb, Jesus. 
	O clement, O loving, O sweet Virgin Mary.
}

\begin{latinenglishsection}

\latinenglish{
	\Vbar. Ora pro nobis, sancta Dei Genitrix.\\
	\Rbar. Ut digni efficamur promissionibus Christi.
	
	Orémus.
	Omnipotens sempiterne Deus, qui gloriosae Virginis Matris Mariae corpus et animam, ut dignum Filii tui habitaculum effici mereretur, Spiritu Sancto cooperante,\\ praeparasti,\GreDagger\ da, ut cuius\\ commemoratio\textit{ne lae}\textbf{ta}mur; eius pia intercessione, ab instantibus malis et a morte perpetua liberemur. 
	Per eundem Christum Dominum nostrum.
	\Rbar.~Amen.
}{
	Vouchsafe that I may praise thee, O sacred Virgin.
	\Rbar.~Give me stength against thine enemies.
	
	Let us pray.
	Almighty and everlasting God, Who by the working of the Holy Spirit didst prepare both body and soul of the glorious Virgin Mother, Mary, that she might deserve to be made a worthy dwelling for Thy Son, grant that we who rejoice in her memory, may, by her loving intercession, be delivered from present evils and from lasting death, through the same Christ our Lord.
	\Rbar.~Amen.
}
\end{latinenglishsection}

\textit{\color{red}The Officiant says the following:}

\begin{latinenglishsection}

\latinenglish{
	\Vbar. Divínum auxílium máneat semper nobíscum.\\
	\Rbar. Amen.
}{
	May the divine assistance remain always with us.
	\Rbar.~Amen.
}
\end{latinenglishsection}

\begin{rubricbox}

After the Office, all \textbf{kneel} and pray in silence for a time.

\end{rubricbox}

\end{document}