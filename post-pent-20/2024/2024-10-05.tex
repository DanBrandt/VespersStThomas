\documentclass{../../vespers-booklet}
\usepackage{multicol}

\begin{document}

% TODO: Update the title for the specific feast
\chapter*{Vespers of the Saturday Preceding the First Sunday of October}

%\section*{Beginning of the Office}

\begin{rubricbox}

{\color{red}When the Officiant kneels, all \textbf{kneel} and pray silently.
Then, when the Officiant stands, all \textbf{stand} and say silently one \textit{Pater noster} (Our Father) and \textit{Ave Maria} (Hail Mary).
Then all make the sign of the cross with the Officiant as he intones:}

\end{rubricbox}

% TODO: Make sure that the tone of the deus adjutorium matches the season primarily and the solemnity of the feast secondarily
 \gresetinitiallines{1}
\gregorioscore{../../common/deus-in-adjutorium}

\textit{
O God, come to my assistance.
{\color{red}\Rbar.}~O Lord, make haste to help me.
Glory be to the Father, and to the Son, and to the Holy Spirit,
as it was in the beginning, is now, and ever shall be, world without end. Amen.
Alleluia.}

\vfill\pagebreak

%TODO: Add that the correct psalms, and verify that their tones, and their associated pointed text are correct

\section*{Psalm 143 (1-8)}

\textit{\textnormal{Ant. 1.} Blessed be the Lord * my support, and my deliverer.
 \textnormal{Ps.} Blessed be the Lord my God, who teacheth my hands to fight, * and my fingers to war.}
 
 \begin{rubricbox}

{\color{red}All remain standing throughout the first antiphon.
After the psalm is intoned by the Cantor, all \textbf{sit} at the asterisk.}

\end{rubricbox}

\gresetinitiallines{1}
\gregorioscore{ps143-1-antiphon}

\gresetinitiallines{0}
\gregorioscore{ps143-1-intonation}

 \begin{latinenglishsection}

\latinenglish{

	2. Misericórdia mea, et refúgi\textit{um} \textbf{me}um:~* suscéptor meus, et libe\textit{rá}\textit{tor} \textbf{me}us:

3. Protéctor meus, et in ipso \textit{spe}\textbf{rá}vi:~* qui subdit pópulum \textit{me}\textit{um} \textbf{sub} me.

4. Dómine, quid est homo quia innotuís\textit{ti} \textbf{e}i?~* aut fílius hóminis, quia ré\textit{pu}\textit{tas} \textbf{e}um?

5. Homo vanitáti sími\textit{lis} \textbf{fac}tus est:~* dies ejus sicut um\textit{bra} \textit{præ}\textbf{tér}eunt.

6. Dómine, inclína cælos tuos, et \textit{de}\textbf{scén}de:~* tange montes, et \textit{fu}\textit{mi}\textbf{gá}bunt.

7. Fúlgura coruscatiónem, et dissipá\textit{bis} \textbf{e}os:~* emítte sagíttas tuas, et contur\textit{bá}\textit{bis} \textbf{e}os.

8. Emítte manum tuam de alto,~{\color{red}\GreDagger} éripe me, et líbera me de a\textit{quis} \textbf{mul}tis:~* de manu filiórum a\textit{li}\textit{e}\textbf{nó}rum.

9. Quorum os locútum est va\textit{ni}\textbf{tá}tem:~* et déxtera eórum, déxtera in\textit{i}\textit{qui}\textbf{tá}tis.

10. {\color{red}\textit{(bow)}} Glória Patri, \textit{et} \textbf{Fí}lio,~* et Spirí\textit{tu}\textit{i} \textbf{Sanc}to.

11. {\color{red}\textit{(rise)}} Sicut erat in princípio, et nunc, \textit{et} \textbf{sem}per,~* et in s\'{\ae}cula sæcu\textit{ló}\textit{rum}. \textbf{A}men.


 %%

}{
	%1.  Blessed be the Lord my God, who teacheth my hands to fight, and my fingers to war.
 
 2. My mercy, and my refuge: my support, and my deliverer.
 
 3. My protector, and I have hoped in him: who subdueth my people under me. 
 
 4. Lord, what is man, that thou art made known to him? or the son of man, that thou makest account of him? 
 
 5. Man is like to vanity: his days pass away like a shadow. 
 
 6. Lord, bow down thy heavens and descend: touch the mountains and they shall smoke.

7. Send forth lightning, and thou shalt scatter them:
shoot out thy arrows, and thou shalt trouble them.

8. Put forth thy hand from on high, take me out, and deliver me from many waters:
from the hand of strange children. 

9. Whose mouth hath spoken vanity: and their right hand is the right hand of iniquity. 

Glory be.
 %%
}

\end{latinenglishsection}

\begin{rubricbox}

{\color{red}The antiphon is repeated: Benedíctus Dóminus\textit{\dots}} %%

\end{rubricbox}

\gresetinitiallines{1}
\gregorioscore{ps143-1-antiphon}

\vfill\pagebreak

%%

\section*{Psalm 143 (9-15)}

\textit{\textnormal{Ant. 2.} Happy is that people * whose God is the Lord.
 \textnormal{Ps.} To thee, O God, I will sing a new canticle: * on the psaltery and an instrument of ten strings I will sing praises to thee.}
 
 \begin{rubricbox}

{\color{red} Only the Cantor stands to intone the antiphon. He then sits at the asterisk. He rises again to intone the Psalm, and again sits at the asterisk. The choir remains seated the entire time.}

\end{rubricbox}

\gresetinitiallines{1}
\gregorioscore{ps143-2-antiphon}

\gresetinitiallines{0}
\gregorioscore{ps143-2-intonation}

 \begin{latinenglishsection}

\latinenglish{

	2. Qui das salútem \textbf{ré}gibus:~* qui redemísti David, servum tuum, de gládio malígno: \textit{é}\textit{ri}\textbf{pe} me.

3. Et érue me de manu filiórum alienórum,~{\color{red}\GreDagger} quorum os locútum est vani\textbf{tá}tem:~* et déxtera eórum, déxtera in\textit{i}\textit{qui}\textbf{tá}tis.

4. Quorum fílii, sicut novéllæ plantati\textbf{ó}nes~* in juven\textit{tú}\textit{te} \textbf{su}a.

5. Fíliæ eórum com\textbf{pó}sitæ:~* circumornátæ ut simili\textit{tú}\textit{do} \textbf{tem}pli.

6. Promptuária eórum \textbf{ple}na:~* eructántia ex \textit{hoc} \textit{in} \textbf{il}lud.

7. Oves eórum fœtósæ, abundántes in egréssibus \textbf{su}is:~* boves e\textit{ó}\textit{rum} \textbf{cras}sæ.

8. Non est ruína macériæ, neque \textbf{tráns}itus:~* neque clamor in platé\textit{is} \textit{e}\textbf{ó}rum.

9. Beátum dixérunt pópulum, cui \textbf{hæc} sunt:~* beátus pópulus, cujus Dóminus \textit{De}\textit{us} \textbf{e}jus.

10. {\color{red}\textit{(bow)}} Glória Patri, et \textbf{Fí}lio,~* et Spirí\textit{tu}\textit{i} \textbf{Sanc}to.

11. {\color{red}\textit{(rise)}} Sicut erat in princípio, et nunc, et \textbf{sem}per,~* et in s\'{\ae}cula sæcu\textit{ló}\textit{rum}. \textbf{A}men.


 %%

}{
	%1. To thee, O God, I will sing a new canticle: on the psaltery and an instrument of ten strings I will sing praises to thee. 

2. Who givest salvation to kings: who hast redeemed thy servant David from the malicious sword: Deliver me.

3. And rescue me out of the hand of strange children; whose mouth hath spoken vanity:
and their right hand is the right hand of iniquity.

4. Whose sons are as new plants in their youth.
 
5. Their daughters decked out, adorned round about after the similitude of a temple. 
 
6. Their storehouses full, flowing out of this into that.
 
7. Their sheep fruitful in young, abounding in their goings forth: their oxen fat.
 
8. There is no breach of wall, nor passage, nor crying out in their streets. 
 
9. They have called the people happy, that hath these things: but happy is that people whose God is the Lord. 

Glory be.
 %%
}

\end{latinenglishsection}

\begin{rubricbox}

{\color{red}The antiphon is repeated: Beátus pópulus\textit{\dots}} %%

\end{rubricbox}

\gresetinitiallines{1}
\gregorioscore{ps143-2-antiphon}

\vfill\pagebreak

%%

\section*{Psalm 144 (1-7)}

\textit{\textnormal{Ant. 3.} For the Lord is great * and exceedingly to be praised: His greatness has no end.
 \textnormal{Ps.} I will extol thee, O God my king: * and I will bless thy name for ever; yea, for ever and ever.}
 
  \begin{rubricbox}

{\color{red} The remaining Psalms are said in the same manner as the second.}

\end{rubricbox}

\gresetinitiallines{1}
\gregorioscore{ps144-1-antiphon}

\gresetinitiallines{0}
\gregorioscore{ps144-1-intonation}

 \begin{latinenglishsection}

\latinenglish{

	2. Per síngulos dies bene\textbf{dí}cam \textbf{ti}bi:~* et laudábo nomen tuum in s\'{\ae}culum, et in s\'{\ae}\textit{cu}\textit{lum} \textbf{s\'{\ae}}culi.

3. Magnus Dóminus, et lau\textbf{dá}bilis \textbf{ni}mis:~* et magnitúdinis ejus \textit{non} \textit{est} \textbf{fi}nis.

4. Generátio et generátio laudábit \textbf{ó}pera \textbf{tu}a:~* et poténtiam tuam pro\textit{nun}\textit{ti}\textbf{á}bunt.

5. Magnificéntiam glóriæ sanctitátis \textbf{tu}æ lo\textbf{quén}tur:~* et mirabília tu\textit{a} \textit{nar}\textbf{rá}bunt.

6. Et virtútem terribílium tu\textbf{ó}rum \textbf{di}cent:~* et magnitúdinem tu\textit{am} \textit{nar}\textbf{rá}bunt.

7. Memóriam abundántiæ suavitátis tuæ \textbf{e}ruc\textbf{tá}bunt:~* et justítia tua \textit{ex}\textit{sul}\textbf{tá}bunt.

8. {\color{red}\textit{(bow)}} Glória \textbf{Pa}tri, et \textbf{Fí}lio,~* et Spirí\textit{tu}\textit{i} \textbf{Sanc}to.

9. {\color{red}\textit{(rise)}} Sicut erat in princípio, et \textbf{nunc}, et \textbf{sem}per,~* et in s\'{\ae}cula sæcu\textit{ló}\textit{rum}. \textbf{A}men.
 %%

}{
	\input{../../psalms/english/ps144.1} %%
}

\end{latinenglishsection}

\begin{rubricbox}

{\color{red}The antiphon is repeated: Magnus Dóminus\textit{\dots}} %%

\end{rubricbox}

\gresetinitiallines{1}
\gregorioscore{ps144-1-antiphon}

\vfill\pagebreak

%%

\section*{Psalm 144 (8-13a)}

\textit{\textnormal{Ant. 4.} The Lord is sweet * to all: and his tender mercies are over all his works.
 \textnormal{Ps.} The Lord is gracious and merciful: * patient and plenteous in mercy.}


\gresetinitiallines{1}
\gregorioscore{ps144-2-antiphon}

\gresetinitiallines{0}
\gregorioscore{ps144-2-intonation}

 \begin{latinenglishsection}

\latinenglish{

	\input{../../psalms/ps144-2-8} %%

}{
	% 1. The Lord is gracious and merciful: patient and plenteous in mercy.

2. The Lord is sweet to all: and his tender mercies are over all his works.

3. Let all thy works, O lord, praise thee: and let thy saints bless thee.

4. They shall speak of the glory of thy kingdom: and shall tell of thy power.

5. To make thy might known to the sons of men: and the glory of the magnificence of thy kingdom.

6. Thy kingdom is a kingdom of all ages: and thy dominion endureth throughout all generations.

Glory be.
 %%
}

\end{latinenglishsection}

\begin{rubricbox}

{\color{red}The antiphon is repeated: Suávis Dóminus\textit{\dots}} %%

\end{rubricbox}

\gresetinitiallines{1}
\gregorioscore{ps144-2-antiphon}

%%

\section*{Psalm 144 (13b-21)}

\textit{\textnormal{Ant. 5.} The Lord is faithful * in all his words: and holy in all his works. 
 \textnormal{Ps.} The Lord is faithful in all his words: and holy in all his works.}


\gresetinitiallines{1}
\gregorioscore{ps144-3-antiphon}

\gresetinitiallines{0}
\gregorioscore{ps144-3-intonation}

 \begin{latinenglishsection}

\latinenglish{

	2. Allevat Dóminus om\textit{nes} \textit{qui} \textbf{cór}ruunt:~* et érigit omnes e\textbf{lí}sos.

3. Oculi ómnium in te \textit{spe}\textit{rant}, \textbf{Dó}mine:~* et tu das escam illórum in témpore oppor\textbf{tú}no.

4. Aperis tu \textit{ma}\textit{num} \textbf{tu}am:~* et imples omne ánimal benedicti\textbf{ó}ne.

5. Justus Dóminus in ómnibus \textit{vi}\textit{is} \textbf{su}is:~* et sanctus in ómnibus opéribus \textbf{su}is.

6. Prope est Dóminus ómnibus invocán\textit{ti}\textit{bus} \textbf{e}um:~* ómnibus invocántibus eum in veri\textbf{tá}te.

7. Voluntátem timéntium se fáciet:~{\color{red}\GreDagger} et deprecatiónem eó\textit{rum} \textit{ex}\textbf{áu}diet:~* et salvos fáciet \textbf{e}os.

8. Custódit Dóminus omnes \textit{di}\textit{li}\textbf{gén}tes se:~* et omnes peccatóres dis\textbf{pér}det.

9. Laudatiónem Dómini loqué\textit{tur} \textit{os} \textbf{me}um:~* et benedícat omnis caro nómini sancto ejus in s\'{\ae}culum, et in s\'{\ae}culum \textbf{s\'{\ae}}culi.

10. {\color{red}\textit{(bow)}} Glória Pa\textit{tri}, \textit{et} \textbf{Fí}lio,~* et Spirítui \textbf{Sanc}to.

11. {\color{red}\textit{(rise)}} Sicut erat in princípio, et \textit{nunc}, \textit{et} \textbf{sem}per,~* et in s\'{\ae}cula sæculórum. \textbf{A}men.

 %%

}{
	%1. The Lord is faithful in all his words: and holy in all his works.

2. The Lord lifteth up all that fall: and setteth up all that are cast down.

3. The eyes of all hope in thee, O Lord: and thou givest them meat in due season.

4. Thou openest thy hand, and fillest with blessing every living creature.

5. The Lord is just in all his ways: and holy in all his works.

6. The Lord is nigh unto all them that call upon him: to all that call upon him in truth.

7. He will do the will of them that fear him: and he will hear their prayer, and save them.

8. The Lord keepeth all them that love him; but all the wicked he will destroy.

9. My mouth shall speak the praise of the Lord: and let all flesh bless thy holy name for ever; yea, for ever and ever. 

Glory be.
 %%
}

\end{latinenglishsection}

\begin{rubricbox}

{\color{red}The antiphon is repeated: Fidélis Dóminus\textit{\dots}} %%

\end{rubricbox}

\gresetinitiallines{1}
\gregorioscore{ps144-3-antiphon}

%%

%TODO: Verify that the little chapter is fitting for the feast

\section*{Little Chapter (Romans 11:33)}

\textit{\color{red}All stand. The Officiant leads the Little Chapter:}

\begin{latinenglishsection}

\latinenglish{
	O Altitúdo divitiárum sapiéntiæ, et sciéntiæ Dei, quam incomprehensibília sunt iudícia eius, et investigábiles viæ eius!\\
	{\color{red}\Rbar.}~Deo grátias.
}{
	O the depth of the riches of the wisdom and of the knowledge of God! How incomprehensible are his judgments, and how unsearchable his ways!\\
	 {\color{red}\Rbar.}~Thanks be to God.
}

\end{latinenglishsection}

\vfill\pagebreak

% TODO: Verify that the hymn is correct for the feast (including the responsory after the hymn)

\section*{Hymn}

\textit{\color{red} The Cantor leads the hymn:}

\gresetinitiallines{1}
\gregorioscore{../../hymns/o-lux-beata-trinitas}

{\itshape

   1.   O Trinity of blessed Light,
	O Unity of sovereign might,
	as now the fiery sun departs,
	shed Thou Thy beams within our hearts.
    
   2.   To Thee our morning song of praise,
	to Thee our evening prayer we raise;
	Thee may our glory evermore
	in lowly reverence adore.
    
    3.  All laud to God the Father be;
	all praise, Eternal Son, to Thee;
	all glory, as is ever meet,
	to God the Holy Paraclete.
	Amen.
}

\textit{\color{red}The Cantor says the following before all reply afterwards:}

\gresetinitiallines{0}
\gabcsnippet{
(c3) <c><sp>V/</sp>.</c> Ves(h)per(h)tí(h)na(h) o(h)rá(h)ti(h)o(h) as(h)cén(h)dat(h) ad(h) te(h), Dó(h)mi(h)ne.(g'_/hvGF'E/fgf.) (::)
}

\gresetinitiallines{0}
\gabcsnippet{
(c3) <c><sp>R/</sp>.</c> Et(h) des(h)cén(h)dat(h) su(h)per(h) nos(h) mi(h)se(h)ri(h)cór(h)di(h)a(h) tu(h)a.(g'_/hvGF'E/fgf.) (::)
}

\textit{{\color{red}\Vbar.}~Let the evening prayer ascend unto thee, O Lord.
{\color{red}\Rbar.}~And let there descend upon us thy mercy.}

%TODO: Verify that the magnificat antiphon is correct and match the magnificat intonation and pointed text with the tone

\vfill\pagebreak

\section*{Magnificat}

\textit{\textnormal{Ant Magn.} The Lord open your hearts * in His law and commandments, and may the Lord our God send peace.
\textnormal{Cant.} My soul doth magnify the Lord: and my spirit hath rejoiced in God my Saviour.}

\begin{rubricbox}

{\color{red}The Cantor leads by intoning the antiphon and the first verse.}

\end{rubricbox}

\gresetinitiallines{1}
\gregorioscore{magnificat-antiphon-only}

\begin{rubricbox}

{\color{red}All \textbf{stand} and make the sign of the cross with the Cantor.}

\end{rubricbox}

\gresetinitiallines{0}
\gregorioscore{magnificat-intonation}

 \begin{latinenglishsection}

\latinenglish{	
3. \textit{Quia} respéxit humilitátem ancíllæ \textbf{su}æ:~* ecce enim ex hoc beátam me dicent omnes gene\textit{ra}\textit{ti}\textbf{ó}nes.

4. \textit{Quia} fecit mihi magna qui \textbf{pot}ens est:~* et sanctum \textit{no}\textit{men} \textbf{e}jus.

5. \textit{Et mi}sericórdia ejus a progénie in pro\textbf{gé}nies~* timén\textit{ti}\textit{bus} \textbf{e}um.

6. \textit{Fecit} poténtiam in bráchio \textbf{su}o:~* dispérsit supérbos mente \textit{cor}\textit{dis} \textbf{su}i.

7. \textit{Depó}suit poténtes de \textbf{se}de,~* et exal\textit{tá}\textit{vit} \textbf{hú}miles.

8. \textit{Esu}riéntes implévit \textbf{bo}nis:~* et dívites dimí\textit{sit} \textit{in}\textbf{á}nes.

9. \textit{Suscé}pit Israël púerum \textbf{su}um,~* recordátus\\ misericór\textit{di}\textit{æ} \textbf{su}æ.

10. \textit{Sicut} locútus est ad patres \textbf{nos}tros,~* Abraham et sémini e\textit{jus} \textit{in} \textbf{s\'{\ae}}cula.

}{	
	\input{../../psalms/english/magnificat}
}

\end{latinenglishsection}

\textit{\color{red}(bow)} \textit{Glóri}a Patri, et \textbf{Fí}lio,~* et Spirí\textit{tu}\textit{i} \textbf{Sanc}to.

\textit{\color{red}(rise)} \textit{Sicut} erat in princípio, et nunc, et \textbf{sem}per,~* et in s\'{\ae}cula sæcu\textit{ló}\textit{rum}. \textbf{A}men.

\begin{rubricbox}

{\color{red}All \textbf{sit} and repeat the antiphon: Adapériat Dóminus\textit{\dots}, %%
then \textbf{stand} for the prayer.}

\end{rubricbox}

\vfill\pagebreak

%TODO: Verify (with the antiphonary) that the collect is proper for the season. If it is not in antiphonary, use the missal for the feast.

\section*{Collect}

\textit{\color{red}The Officiant leads the collect:}

\begin{latinenglishsection}

\latinenglish{
	{\color{red}\Vbar.}~Dómine exáudi oratiónem meam.\\
	{\color{red}\Rbar.}~Et clamor meus ad te véniat.
	
	Orémus.
	Largíre, qu\'{\ae}sumus, Dómine, fidélibus tuis indulgéntiam placátus et pacem: ut páriter ab ómnibus mundéntur offénsis, et secúra tibi mente desérviant.\\
	Per Dóminum nostrum Iesum Christum, Fílium tuum: qui tecum vivit et regnat in unitáte Spíritus Sancti, Deus, per ómnia s\'{\ae}cula sæculórum.\\
	{\color{red}\Rbar.}~Amen.
}{
	{\color{red}\Vbar.} Lord, hear my prayer.
	{\color{red}\Rbar.}~And let my cry come unto Thee.
	
	Let us pray.
	Grant, we beseech thee, O Lord, to thy faithful people pardon and peace, that they may be cleansed from all their sins, and serve thee with a quiet mind.\\
	Through Jesus Christ, thy Son our Lord, Who liveth and reigneth with thee, in the unity of the Holy Ghost, God, world without end.\\
	{\color{red}\Rbar.}~Amen.
}
\end{latinenglishsection}

%TODO: Add commemorations for the date

%\textit{\color{red}For commemorations, the Cantor intones the antiphon and says the responsorial prayer afterwards. The Officiant prays the associated collect.}
%
%\textit{\color{red}The Officiant leads the following:}
%
%\begin{latinenglishsection}
%
%\latinenglish{
%	{\color{red}\Vbar.}~Dómine exáudi oratiónem meam.\\
%	{\color{red}\Rbar.}~Et clamor meus ad te véniat.
%}{
%	{\color{red}\Vbar.}~Lord, hear my prayer. {\color{red}\Rbar.}~And let my cry come unto Thee.
%	{\color{red}\Vbar.}~Let us bless the Lord. {\color{red}\Rbar.}~Thanks be to God.
%}
%
%\end{latinenglishsection}
%
%\vfill\pagebreak

\textit{\color{red}The Cantor leads the Benedicamus:}

\gresetinitiallines{1}
\gregorioscore{../../common/benedicamus_sundays_per_annum}

\textit{\color{red}The Officiant leads the following:}

\begin{latinenglishsection}

\latinenglish{
	{\color{red}\Vbar.} Fidélium ánimæ, per misericórdiam Dei, requiéscant in pace. \\
	{\color{red}\Rbar.} Amen.
}{
	May the souls of the faithful departed, through the mercy of God, rest in peace. {\color{red}\Rbar.}~Amen.
}

\end{latinenglishsection}

\begin{rubricbox}

{\color{red} After the Office, all \textbf{kneel} and pray in silence for a time.}

\end{rubricbox}

\end{document}