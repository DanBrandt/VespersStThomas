\documentclass{../vespers-booklet}
\usepackage{multicol}

\begin{document}

% TODO: Update the title for the specific feast
\chapter*{Second Vespers of the Ascension of Our Lord}

\begin{center}
\includegraphics[width=0.7\linewidth]{ascension_line_art}
\end{center}

\vfill\newpage

%\section*{Beginning of the Office}

\begin{rubricbox}

{\color{red}When the Officiant kneels, all \textbf{kneel} and pray silently.
Then, when the Officiant stands, all \textbf{stand} and say silently one \textit{Pater noster} (Our Father) and \textit{Ave Maria} (Hail Mary).
Then all make the sign of the cross with the Officiant as he intones:}

\end{rubricbox}

% TODO: Make sure that the tone of the deus adjutorium matches the season primarily and the solemnity of the feast secondarily
 \gresetinitiallines{1}
\gregorioscore{../common/deus-in-adjutorium-solemn}

\textit{
O God, come to my assistance.
{\color{red}\Vbar.}~O Lord, make haste to help me.
Glory be to the Father, and to the Son, and to the Holy Spirit,
as it was in the beginning, is now, and ever shall be, world without end. Amen.
Praise to Thee, O Lord, King of endless glory.}

%TODO: Add that the correct psalms, and verify that their tones, and their associated pointed text are correct

\section*{Psalm 109}

\textit{\textnormal{Ant. 1.} Men of Galilee, why do you stand looking up to heaven? This Jesus who has been take up from you into heaven, will come in the same way as you have seen Him going up to heaven, alleluia.
 \textnormal{Ps.} The Lord said to my Lord...}
 
 \begin{rubricbox}

{\color{red}All remain standing throughout the first antiphon.
After the psalm is intoned by the Cantor, all \textbf{sit} at the asterisk.}

\end{rubricbox}

\gresetinitiallines{1}
\gregorioscore{ps109-antiphon}

\gresetinitiallines{0}
\gregorioscore{ps109-intonation}

 \begin{latinenglishsection}

\latinenglish{

	\input{../psalms/ps109-7} %%

}{
	\input{../psalms/english/ps109} %%
}

\end{latinenglishsection}

\begin{rubricbox}

{\color{red}The antiphon is repeated: Viri Galilei\textit{\dots}} %%

\end{rubricbox}

\section*{Psalm 110}

\textit{\textnormal{Ant. 2.} While they were beholding Him going up to heaven, they said, alleluia.
 \textnormal{Ps.} I praise Thee, O Lord, with all my heart...}

\gresetinitiallines{1}
\gregorioscore{ps110-antiphon}

\gresetinitiallines{0}
\gregorioscore{ps110-intonation}

 \begin{latinenglishsection}

\latinenglish{

	\input{../psalms/ps110-8GSTAR} %%

}{
	\input{../psalms/english/ps110} %%
}

\end{latinenglishsection}

\begin{rubricbox}

{\color{red}The antiphon is repeated: Cumque intueréntur\textit{\dots}} %%

\end{rubricbox}

\section*{Psalm 111}

\textit{\textnormal{Ant. 3.} Lifting up his hands, He blessed them and was carried up to heaen, alleluia.
 \textnormal{Ps.} Blessed is the man that feareth the Lord...}

\gresetinitiallines{1}
\gregorioscore{ps111-antiphon}

\gresetinitiallines{0}
\gregorioscore{ps111-intonation}

 \begin{latinenglishsection}

\latinenglish{

	\input{../psalms/ps111-4ASTAR} %%

}{
	\input{../psalms/english/ps111} %%
}

\end{latinenglishsection}

\begin{rubricbox}

{\color{red}The antiphon is repeated: Elevátis mánibus\textit{\dots}} %%

\end{rubricbox}

\vfill\pagebreak

\section*{Psalm 112}

\textit{\textnormal{Ant. 4.} While they were looking on, He was raised up and a cloud received Him into heaven, alleluia.
 \textnormal{Ps.} Praise the Lord, ye children...}

\gresetinitiallines{1}
\gregorioscore{ps112-antiphon}

\gresetinitiallines{0}
\gregorioscore{ps112-intonation}

 \begin{latinenglishsection}

\latinenglish{

	\input{../psalms/ps112-8G} %%

}{
	\input{../psalms/english/ps112} %%
}

\end{latinenglishsection}

\begin{rubricbox}

{\color{red}The antiphon is repeated: Vidéntibus illis\textit{\dots}} %%

\end{rubricbox}


%TODO: Verify that the little chapter is fitting for the feast

\section*{Chapter (Acts 1, 1-2)}

\textit{\color{red}The Officiant leads the Little Chapter:}

\begin{latinenglishsection}

\latinenglish{
	Primum quiden sermónem feci dei ómnibus, o Theóphile, qu\ae c\oe pit Iesus fácere et docére {\color{red}\GreDagger} usque in diem, qua pr\ae cipiens Apóstolis per Spiritum Sanctum, quos elégit, {\color{red}*} assumptus est.
}{
	In the former book, O Theophilus, I spoke of all that Jesus did and taught from the beginning, until the day on which He was taken up, after He had given commandments through the Holy Spirit to the Apostles, whom He had chosen.
}

\end{latinenglishsection}

\subsection*{Short Responsory}

\gresetinitiallines{0}
\gregorioscore{short_responsory}
\textit{{\color{red}\Vbar.} When Christ ascended up on high. * Alleluia, alleluia.
{\color{red}\Rbar.} When Christ ascended up on high. * Alleluia, alleluia.
{\color{red}\Vbar.} He led captivity captive. {\color{red}\Rbar.} Alleluia, alleluia.
{\color{red}\Vbar.} Glory be to the Father, and to the Son, * and to the Holy Ghost.
{\color{red}\Rbar.} When Christ ascended up on high. * Alleluia, alleluia.}

% TODO: Verify that the hymn is correct for the feast (including the responsory after the hymn)

\section*{Hymn}

\textit{\color{red}The Cantor leads the hymn:}

\gresetinitiallines{1}
\gregorioscore{../hymns/iesu_nostra_redemptio}

{\itshape

   1.  Jesus, Redemption all divine,
   Whom her we love, for Whom we pine,
   God, working out creation's plan
   And in the latter time made Man;
    
   2.  What love of Thine was that, which led
   To take our woes upon Thy head
   And pangs and cruel death to bear
   To ransomus from death's despair.
    
    3. To Thee hell's gate gave ready way,
    Demanding there his captive prey;
    And now in pomp and victor's pride
    Thou sittest at They Father's side.
    
   4.  Let very mercy force Thee still
   To spare us, conquering all our ill;
   And, granting what we ask, on high
   With Thine own face to satisfy.
    
    5. Be Thou our joy and Thou our guard,
    Who art to be our great reward;
    Our glory and our boast in Thee
    Forever and forever be.
    Amen.
}

\textit{\color{red}The Cantor says the following before all reply afterwards:}

\gresetinitiallines{0}
\gabcsnippet{
(c3) <c><sp>V/</sp>.</c> A(h)scen(h)dit(h) De(h)us(h) in(h) ju(h)bi(h)la(h)ti(h)o(hi)ne(h_'),(,) al(h)le(fe)lu(fh_)i(hi)a(H'Ghih.ghG'FEfgge.). (::)
}

\gresetinitiallines{0}
\gabcsnippet{
(c3) <c><sp>R/</sp>.</c> Et(h) Do(h)mi(h)nus(h) in(h) vo(h)ce(h) tu(hi)bæ(h_'),(,)  al(h)le(fe)lu(fh_)i(hi)a(H'Ghih.ghG'FEfgge.). (::)
}

\textit{{\color{red}\Vbar.}~God ascends amid shouts of triumph, alleluia.
{\color{red}\Rbar.}~And the Lord with the sound of trumpets, alleluia.}

%TODO: Verify that the magnificat antiphon is correct and match the mangificat intonation and pointed text with the tone

\section*{Magnificat}

\textit{\textnormal{Ant Magn.}  O King of glory, Lord of hosts, * Who hast this day exalted thine Own Self, with great triumph, above all the heavens, leave us not orphans but send unto us the Promise of the Father, even the Spirit of truth, alleluia.
\textnormal{Cant.} My soul doth magnify the Lord: and my spirit hath rejoiced in God my Saviour...}

\begin{rubricbox}

{\color{red}The Cantor leads by intoning the antiphon and the first verse.}

\end{rubricbox}

\gresetinitiallines{1}
\gregorioscore{magnificat-antiphon-only}

\begin{rubricbox}

{\color{red}All \textbf{stand} and make the sign of the cross with the Cantor.}

\end{rubricbox}

\gresetinitiallines{0}
\gregorioscore{magnificat-intonation}

 \begin{latinenglishsection}

\latinenglish{	
3. Quia respéxit humilitátem ancíllæ \textbf{su}æ:~* 
ecce enim ex hoc beátam me dicent omnes genera\textit{ti}\textbf{ó}nes.

4. Quia fecit mihi magna qui \textbf{pot}ens est:~* 
et sanctum no\textit{men} \textbf{e}jus.

5. Et misericórdia ejus a progénie in pro\textbf{gé}nies~* 
timénti\textit{bus} \textbf{e}um.

6. Fecit poténtiam in bráchio \textbf{su}o:~* 
dispérsit supérbos mente cor\textit{dis} \textbf{su}i.

7. Depósuit poténtes de \textbf{se}de,~* 
et exaltá\textit{vit} \textbf{hú}\textbf{mi}les.

8. Esuriéntes implévit \textbf{bo}nis:~* 
et dívites dimísit \textit{in}\textbf{á}nes.

9. Suscépit Israël púerum \textbf{su}um,~* 
recordátus misericórdi\textit{æ} \textbf{su}æ.

10. Sicut locútus est ad patres \textbf{nos}tros,~* 
Abraham et sémini ejus \textit{in} \textbf{s\'{\ae}}\textbf{cu}la.
}{	
	\input{../psalms/english/magnificat}
}

\end{latinenglishsection}

\textit{\color{red}(bow)} Glória Patri, et \textbf{Fí}lio,~* 
et Spirítu\textit{i} \textbf{Sanc}to.

\textit{\color{red}(rise)} Sicut erat in princípio, et nunc, et \textbf{sem}per,~*
et in s\'{\ae}cula sæculó\textit{rum}. \textbf{A}men.

\begin{rubricbox}

{\color{red}All \textbf{remain standing} and repeat the antiphon: O Rex gloriae\textit{\dots}, %%
then \textbf{stand} for the prayer.}

\end{rubricbox}

\gresetinitiallines{1}
\gregorioscore{../common/kyrie_eleison_simplex}

%TODO: Add commemorations for the date

%\textit{\color{red}For commemorations, the Cantor intones the antiphon and says the responsorial prayer afterwards. The Officiant prays the associated collect.}
%
%\gresetinitiallines{1}
%\gregorioscore{commemoration}
%

%\vfill\pagebreak

\textit{\color{red}The Officiant leads the following, saying the Pater Noster aloud:}

\begin{latinenglishsection}

\latinenglish{
	Pater noster, qui es in cælis, sanctificétur nomen tuum: advéniat regnum tuum: fiat volúntas tua, sicut in cælo et in terra. Panem nostrum cotidiánum da nobis hódie: et dimítte nobis débita nostra, sicut et nos dimíttimus debitóribus nostris:\\
	{\color{red}\Vbar.} Et ne nos indúcas in tentatiónem:
	
	{\color{red}\Rbar.} Sed líbera nos a malo.
}{
	Our Father, who art in heaven, Hallowed be thy name. Thy kingdom come. Thy will be done on earth as it is in heaven. Give us this day our daily bread. And forgive us our trespasses, as we forgive those who trespass against us.\\
	{\color{red}\Vbar.} And lead us not into temptation: 
	
	{\color{red}\Rbar.} But deliver us from evil.
}

\end{latinenglishsection}

\vfill\pagebreak

%TODO: Verify (with the antiphonary) that the collect is proper for the season. If it is not in antiphonary, use the missal for the feast.

\section*{Collect}

\textit{\color{red}The Officiant leads the collect:}

\begin{latinenglishsection}

\latinenglish{
	{\color{red}\Vbar.}~Dómine exáudi oratiónem meam.\\
	{\color{red}\Rbar.}~Et clamor meus ad te véniat.
	
	Orémus.
	Concéde, qu\'{\ae}sumus, omnípotens Deus: ut, qui hodiérna die Unigénitum tuum, Redemptórem nostrum, ad cælos ascendísse crédimus; ipsi quoque mente in cæléstibus habitémus.\\
	Per eúndem Dóminum nostrum Iesum Christum Fílium tuum, qui tecum vivit et regnat in unitáte Spíritus Sancti, Deus, per ómnia s\'{\ae}cula sæculórum.\\
	{\color{red}\Rbar.}~Amen.
}{
	{\color{red}\Vbar.} O Lord, hear my prayer.
	{\color{red}\Rbar.}~And let my cry come unto Thee.
	
	Let us pray.
	Grant, we beseech thee, Almighty God, that just as we do believe thine Only-Begotten Son, our Saviour, to have this day ascended into the heavens, so we may also in heart and mind thither ascend, and with Him continually dwell.\\
	Through the same Jesus Christ, thy Son, Our Lord, Who liveth and reigneth with thee in the unity of the Holy Ghost, God, world without end. \\
	{\color{red}\Rbar.}~Amen.
}
\end{latinenglishsection}

%TODO: Add the Marian Anthem for the season and verify that the oration afterwards is correct

%\section*{Marian Anthem}
%
%\textit{\color{red}The Cantor leads the Marian anthem and responses afterwards; the Officiant leads the ending collect:}
%
%\gresetinitiallines{1}
%\gregorioscore{../marian-anthems/regina-caeli-solemn}
%
%{\itshape
%	Queen of heaven, rejoice, alleluia; 
%	for He whom thou was chosen to bear, alleluia; 
%	has risen as He said, alleluia; 
%	pray for us to God, alleluia.
%}
%
%\begin{latinenglishsection}
%
%\latinenglish{
%	{\color{red}\Vbar.} Gaude et laetare, Virgo Maria, alleluia. \\
%	{\color{red}\Rbar.} Quia surrexit Dominus vere, alleluia.
%	
%	Orémus.
%	 Deus, qui per resurrectionem Filii tui, Domini nostri Iesu Christi, mundum laetificare dignatus es: praesta, quaesumus; ut per eius Genetricem Virginem Mariam, perpetuae capiamus gaudia vitae. 
%	 Per eundem Christum Dominum nostrum. Amen.
%	{\color{red}\Rbar.}~Amen.
%}{
%	{\color{red}\Vbar.}Rejoice and be glad, O Virgin Mary, alleluia.
%	{\color{red}\Rbar.}~For the Lord has truly risen, alleluia.
%	
%	Let us pray. 
%	O God, who gave joy to the world through the resurrection of Thy Son, our Lord Jesus Christ, grant we beseech Thee, that through the intercession of the Virgin Mary, His Mother, we may obtain the joys of everlasting life. 
%	Through the same Christ our Lord. Amen.
%	{\color{red}\Rbar.}~Amen.
%}
%\end{latinenglishsection}

\textit{\color{red}The Officiant leads the following:}

\begin{latinenglishsection}

\latinenglish{
	{\color{red}\Vbar.} Dómine, exáudi oratiónem meam.
	{\color{red}\Rbar.} Et clamor meus ad te véniat.
}{
	{\color{red}\Vbar.} O Lord, hear my prayer.
	{\color{red}\Rbar.} And let my cry come unto thee.
}

\end{latinenglishsection}

\textit{\color{red}The Cantor leads the Benedicamus:}

\gresetinitiallines{1}
\gregorioscore{../common/benedicamus-2v-solem}

\begin{latinenglishsection}	

\latinenglish{
	{\color{red}\Vbar.} Fidélium ánimæ per misericórdiam Dei requiéscant in pace.\\
	{\color{red}\Rbar.} Amen.

	{\color{red}\Vbar.} Divínum auxílium {\color{red}\maltese} máneat semper nobíscum.\\
	{\color{red}\Rbar.} Amen.
}{
	{\color{red}\Vbar.} May the souls of the faithful, through the mercy of God, rest in peace.
	{\color{red}\Rbar.} Amen.
	
	{\color{red}\Vbar.} May the divine assistance remain always with us.
	{\color{red}\Rbar.}~Amen.
}
\end{latinenglishsection}

\begin{rubricbox}

{\color{red} After the Office, all \textbf{kneel} and pray in silence for a time.}

\end{rubricbox}

\end{document}