\documentclass{../../vespers-booklet}
\usepackage{multicol}
\usepackage{graphics}

\begin{document}

\chapter*{Vespers on Friday}

\begin{center}
\includegraphics[width=\linewidth]{Trinity_With_Sacred_Heart.jpg}
\end{center}

\vfill\pagebreak

\begin{rubricbox}

{\color{red}When the leader kneels, all \textbf{kneel} and pray silently one \textit{Pater noster} (Our Father) and one \textit{Ave Maria} (Hail Mary). When the leader stands, all rise with him.
Then all make the Sign of the Cross with the Officiant as he intones:}

\end{rubricbox}

 \gresetinitiallines{1}
\gregorioscore{../../common/deus-in-adjutorium}

\begin{rubricbox}

{\color{red}After Septuagesima Sunday until Easter, \textit{Alleluia} is replaced with}

\end{rubricbox}

\gresetinitiallines{0}
\gabcsnippet{
	(c3) Laus(h) ti(h)bi(h) Dó(h)mi(h)ne(h'_) Rex(h) æ(h)tér(h)næ(i) gló(h')ri(h)æ.(g.) (::)
}

\textit{
O God, come to my assistance.
{\color{red}\Vbar.}~O Lord, make haste to help me.
Glory be to the Father, and to the Son, and to the Holy Spirit,
as it was in the beginning, is now, and ever shall be, world without end. Amen.
Alleluia. \textnormal{Or:} Praise to Thee, O Lord, King of endless glory.}

\vfill\pagebreak

\section*{Psalm 137}

\textit{\textnormal{Ant. 1.} In the sight\dots \textnormal{Ps.} I will praise Thee, O Lord, with my whole heart: for Thou hast heard the words of my mouth.}

\begin{rubricbox}

{\color{red}All \textbf{stand}. The Cantor intones the antiphon and leads the Psalm through the first half of the first verse (to the asterisk), then all \textbf{sit}, with the Cantor's side finishing the verse. Each side then alternates chanting the verses.}

\end{rubricbox}

\gresetinitiallines{1}
\gregorioscore{ps137-intonation}

 \begin{latinenglishsection}

\latinenglish{
	2. In conspéctu Angelórum psallam \textbf{ti}bi:~* adorábo ad templum sanctum tuum, et confitébor \textbf{nó}mini \textbf{tu}o.

	3. Super misericórdia tua, et veritáte \textbf{tu}a:~* quóniam magnificásti super omne, nomen \textbf{sanc}tum \textbf{tu}um.

	4. In quacúmque die invocávero te, ex\textbf{áu}di me:~* multiplicábis in ánima \textbf{me}a vir\textbf{tú}tem.

	5. Confiteántur tibi, Dómine, omnes reges \textbf{ter}ræ:~* quia audiérunt ómnia verba \textbf{o}ris \textbf{tu}i.

	6. Et cantent in viis \textbf{Dó}mini:~* quóniam magna est \textbf{gló}ria \textbf{Dó}mini.

	7. Quóniam excélsus Dóminus, et humília \textbf{ré}spicit:~* et alta a \textbf{lon}ge co\textbf{gnó}scit.
	
}{
	 2. I will sing praises to Thee in the sight of Thy angels: I will worship towards Thy holy temple, and I will give glory to Thy name.
	 
	 3. For Thy mercy, and for Thy truth: for thou hast magnified Thy holy name above all.
	 
	 4.  In what day soever I shall call upon thee, hear me: Thou shall multiply strength in my soul.

	 5. May all the kings of the earth give glory to Thee: for they have heard all the words of Thy mouth.
	 
	 6. And let them sing in the ways of the Lord: for great is the glory of the Lord.
	 
	 7. For the Lord is high, and looketh on the low: and the high He knoweth afar off.
}

\vfill\pagebreak
	
\latinenglish{
	8. Si ambulávero in médio\\ tribulatiónis, vivificábis me:~{\color{red}\GreDagger}\ et super iram inimicórum meórum extendísti manum \textbf{tu}am,~* et salvum me fecit \textbf{déx}tera \textbf{tu}a.

	9. Dóminus retríbuet pro me:~{\color{red}\GreDagger}\ Dómine, misericórdia tua in\\ \textbf{s\'{\ae}}culum:\\ ~* {\color{red}\textit{(stand)}} ópera mánuum tuárum \textbf{ne} de\textbf{spí}cias.

	10. {\color{red}\textit{(bow)}} Glória Patri, et \textbf{Fí}lio,~* et Spi\textbf{rí}tui \textbf{Sanc}to.

	11.  {\color{red}\textit{(rise)}} Sicut erat in princípio, et nunc, et \textbf{sem}per,~* et in s\'{\ae}cula sæcu\textbf{ló}rum. \textbf{A}men.
}{	 
	 8. If I shall walk in the midst of tribulation, Thou wilt quicken me: and Thou hast stretched forth Thy hand against the wrath of my enemies: and thy right hand hath saved me.
	 
	 9. The Lord will repay for me: Thy mercy, O Lord, endureth for ever: O despise not the work of Thy hands.
	
	10. Glory be to the Father, and to the Son, and to the Holy Spirit.
	
	11. As it was in the beginning, is now, and ever shall be, world without end. Amen.
}

\end{latinenglishsection}

\gresetinitiallines{0}
\gregorioscore{ps137-antiphon}
\textit{\textnormal{Ant.} In the sight of Thy angels, I will sing praises to Thee, my God.}

\section*{Psalm 138}

\textit{\textnormal{Ant. 2.}  \textnormal{Ps.} .}

\begin{rubricbox}

{\color{red}All \textbf{remain standing}. As in the preceding Psalm, the cantor intones the antiphon and leads the first verse to the asterisk, after which all sit, and the Cantor's side finishes the latter half of the verse. Each side alternates chanting the verses. This manner of chanting of the Psalms is continued for the remaining Psalms.}

\end{rubricbox}



\end{document}