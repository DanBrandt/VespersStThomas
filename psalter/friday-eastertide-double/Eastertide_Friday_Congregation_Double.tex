\documentclass[12pt]{../../vespers-sheet}
\usepackage{multicol}

\begin{document}

\chapter*{Vespers on Fridays in Eastertide\\ for Double Feasts}

\begin{center}
\includegraphics[width=\linewidth]{Resurrection.jpg}
\end{center}

\vfill\pagebreak

\begin{rubricbox}

{\color{red}When the leader kneels, all \textbf{kneel} and pray silently one \textit{Pater noster} (Our Father) and \textit{Ave Maria} (Hail Mary). All join the leader as he rises, and all make the sign of the cross with the leader as he intones:}

\end{rubricbox}

 \gresetinitiallines{1}
\gregorioscore{../../common/deus-in-adjutorium}

\textit{
O God, come to my assistance.
{\color{red}\Vbar.}~O Lord, make haste to help me.
Glory be to the Father, and to the Son, and to the Holy Spirit,
as it was in the beginning, is now, and ever shall be, world without end. Amen.
Alleluia. \textnormal{Or:} Praise to Thee, O Lord, King of endless glory.}

\vfill\pagebreak

\begin{rubricbox}

{\color{red}\textbf{All stand and remain standing} while the cantor intones the antiphon to the first asterisk. Then all sit while the cantor remains standing and intones the first verse of the Psalm, after which he sits as the choir continues the remainder of the Psalm.}

\end{rubricbox}

%\vfill\pagebreak

\section*{Psalm 138 (1-13)}

\textit{\textnormal{Ant.} Alleluia, alleluia, alleluia. \textnormal{Ps.} Lord, thou hast proved me, and known me: thou hast known my sitting down, and my rising up.}

\gresetinitiallines{0}
\gregorioscore{pss-antiphon1}

{\color{red}The choir sits. The cantor remains standing while intoning the Psalm.}

\gresetinitiallines{0}
\gregorioscore{ps138-1-13-intonation}

 \begin{latinenglishsection}

\latinenglish{
	\input{../../psalms/ps138-1-13-3g2}	
}{
	2. Thou hast understood my thoughts afar off: my path and my line thou hast searched out
 	
3. And thou hast foreseen all my ways: for there is no speech in my tongue
 	
4. Behold, O Lord, thou hast known all things, the last and those of old: thou hast formed me, and hast laid thy hand upon me.
 	
5. Thy knowledge is become wonderful to me: it is high, and I cannot reach to it.
 	
6. Whither shall I go from thy spirit? or whither shall I flee from thy face?
 	
7. If I ascend into heaven, thou art there: if I descend into hell, thou art present.
 	
8. If I take my wings early in the morning, and dwell in the uttermost parts of the sea:
 	
9. Even there also shall thy hand lead me: and thy right hand shall hold me.
 	
10. And I said: Perhaps darkness shall cover me: and night shall be my light in my pleasures.
 	
11. But darkness shall not be dark to thee, and night shall be light as day: the darkness thereof, and the light thereof are alike to thee.
 	
12. For thou hast possessed my reins: thou hast protected me from my mother's womb.
 	
13. I will praise thee, for thou art fearfully magnified: wonderful are thy works, and my soul knoweth right well.
 	
14 Glory be to the Father, and to the Son, and to the Holy Spirit.
 	
15. As it was in the beginning, is now, and ever shall be, world without end. Amen.

}

\end{latinenglishsection}

%\vfill\pagebreak

\section*{Psalm 138 (14-24)}

\textit{\textnormal{Ps.} I will praise thee, for thou art fearfully magnified: wonderful are thy works, and my soul knoweth right well.}

\gresetinitiallines{0}
\gregorioscore{ps138-14-24-intonation}

 \begin{latinenglishsection}

\latinenglish{
		2. Non est occultátum os meum a te, quod fecísti \textbf{in} oc\textbf{cúl}to:~* et substántia mea in inferi\textit{ó}\textit{ri}\textit{bus} \textbf{ter}ræ.

	3. Imperféctum meum vidérunt óculi tui,~{\color{red}\GreDagger} et in libro tuo \textbf{om}nes scri\textbf{bén}tur:~* dies formabúntur, et \textit{ne}\textit{mo} \textit{in} \textbf{e}is.

	4. Mihi autem nimis honorificáti sunt amíci \textbf{tu}i, \textbf{De}us:~* nimis confortátus est princi\textit{pá}\textit{tus} \textit{e}\textbf{ó}rum.

	5. Dinumerábo eos, et super arénam\\ multi\textbf{pli}ca\textbf{bún}tur:~* exsurréxi, et \textit{ad}\textit{huc} \textit{sum} \textbf{te}cum.

	6. Si occíderis, Deus, \textbf{pec}ca\textbf{tó}res:~* viri sánguinum, de\textit{cli}\textit{ná}\textit{te} \textbf{a} me.

	7. Quia dícitis in cogi\textbf{ta}ti\textbf{ó}ne:~* Accípient in vanitáte ci\textit{vi}\textit{tá}\textit{tes} \textbf{tu}as.

	8. Nonne qui odérunt te, \textbf{Dó}mine, \textbf{ó}\textbf{de}ram?~* et super inimícos tu\textit{os} \textit{ta}\textit{be}\textbf{scé}bam?

	9. Perfécto ódio \textbf{ó}deram \textbf{il}los:~* et inimíci \textit{fac}\textit{ti} \textit{sunt} \textbf{mi}hi.

	10. Proba me, Deus, et \textbf{sci}to cor \textbf{me}um:~* intérroga me, et cognósce \textit{sé}\textit{mi}\textit{tas} \textbf{me}as.

	11. Et vide, si via iniqui\textbf{tá}tis in \textbf{me} est:~* et deduc me in \textit{vi}\textit{a} \textit{æ}\textbf{tér}na.

	12. {\color{red}\textit{(bow while seated)}} Glória \textbf{Pa}tri, et \textbf{Fí}\textbf{li}o,~* et Spi\textit{rí}\textit{tu}\textit{i} \textbf{Sanc}to.

	13. {\color{red}\textit{(rise, remaining seated)}} Sicut erat in princípio, et \textbf{nunc}, et \textbf{sem}per,~* et in s\'{\ae}cula sæ\textit{cu}\textit{ló}\textit{rum}. \textbf{A}men.	
}{
	2. My bone is not hidden from thee, which thou hast made in secret: and my substance in the lower parts of the earth.
 	
3. Thy eyes did see my imperfect being, and in thy book all shall be written: days shall be formed, and no one in them.
 	
4.  But to me thy friends, O God, are made exceedingly honourable: their principality is exceedingly strengthened.
 	
5.  I will number them, and they shall be multiplied above the sand: I rose up and am still with thee.
 	
6. If thou wilt kill the wicked, O God: ye men of blood, depart from me:
 	
7. Because you say in thought: They shall receive thy cities in vain.
 	
8. Have I not hated them, O Lord, that hated thee: and pine away because of thy enemies?

9. I have hated them with a perfect hatred: and they are become enemies to me.
 	
10. Prove me, O God, and know my heart: examine me, and know my paths.
 	
11. And see if there be in me the way of iniquity: and lead me in the eternal way.	
 	
12. Glory be to the Father, and to the Son, and to the Holy Spirit.
 	
13. As it was in the beginning, is now, and ever shall be, world without end. Amen.

}

\end{latinenglishsection}

\section*{Psalm 139}

\textit{\textnormal{Ps.} Deliver me, O Lord, from the evil man: rescue me from the unjust man.}

\gresetinitiallines{0}
\gregorioscore{ps139-intonation}

 \begin{latinenglishsection}

\latinenglish{
	\input{../../psalms/ps139-3g2}	
}{
	\input{../../psalms/english/ps139}
}

\end{latinenglishsection}

\section*{Psalm 140}

\textit{\textnormal{Ps.} I have cried to Thee, O Lord: hear me: hearken to my voice, when I cry to Thee.}

\gresetinitiallines{0}
\gregorioscore{ps140-intonation}

 \begin{latinenglishsection}

\latinenglish{
	\input{../../psalms/ps140-3g2}	
}{
	2. Let my prayer be directed as incense in thy sight; the lifting up of my hands, as evening sacrifice.
	
3. Set a watch, O Lord, before my mouth: and a door round about my lips.

4. Incline not my heart to evil words; to make excuses in sins. 

5. With men that work iniquity: and I will not communicate with the choicest of them.

6. The just shall correct me in mercy, and shall reprove me: but let not the oil of the sinner fatten my head. 

7. For my prayer also shall still be against the things with which they are well pleased: Their judges falling upon the rock have been swallowed up. 

8. They shall hear my words, for they have prevailed: As when the thickness of the earth is broken up upon the ground: 

9. Our bones are scattered by the side of Hell. But o to thee, O Lord, Lord, are my eyes: in thee have I put my trust, take not away my soul.

10. Keep me from the snare, which they have laid for me, and from the stumbling blocks of them that work iniquity.

11. The wicked shall fall in his net: I am alone until I pass.

12. Glory be to the Father, and to the Son, and to the Holy Spirit.
 	
13. As it was in the beginning, is now, and ever shall be, world without end. Amen.
}

\end{latinenglishsection}

\section*{Psalm 141}

\textit{\textnormal{Ps.} I cried to the Lord with my voice: with my voice I made supplication to the Lord.}

\gresetinitiallines{0}
\gregorioscore{ps141-intonation}

 \begin{latinenglishsection}

\latinenglish{
	\input{../../psalms/ps141-3g2}	
}{
	2. In his sight I pour out my prayer, and before him I declare my trouble:
	
3. When my spirit failed me, then thou newest my paths. 

4. In this way wherein I walked, they have hidden a snare for me.

5. I looked on my right hand, and beheld, and there was no one that would know me. 

6. Flight hath failed me: and there is no one that hath regard to my soul.

7. I cried to thee, O Lord: I said: Thou art my hope, my portion in the land of the living.

8. Attend to my supplication: for I am brought very low. 

9. Deliver me from my persecutors; for they are stronger than I.

10. Bring my soul out of prison, that I may praise thy name: the just wait for me, until thou reward me.

11. Glory be to the Father, and to the Son, and to the Holy Spirit.
 	
12. As it was in the beginning, is now, and ever shall be, world without end. Amen.
}

\end{latinenglishsection}

%{\color{red}\textit{All chant the Alleluia antiphon together:}}

\gresetinitiallines{0}
\gregorioscore{pss-antiphon2}
%\textit{\textnormal{Ant.} Alleluia, alleluia, alleluia.}


\begin{rubricbox}

{\color{red}\textbf{All stand.} Continue to the insert for the day for the Little Chapter, Hymn, Magnificat, Collect, any Commemorations, and the Conclusion.}

\end{rubricbox}

\end{document}