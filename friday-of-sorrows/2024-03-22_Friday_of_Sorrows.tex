\documentclass[12pt]{../vespers-sheet} %booklet
\usepackage{multicol}

\begin{document}

% TODO: Update the title for the specific feast
\chapter*{Second Vespers of the Friday of Sorrows}

\begin{center}
\includegraphics[width=0.55\textwidth]{OLSorrows2W}
\end{center}

\vfill\pagebreak

%\section*{Beginning of the Office}

\begin{rubricbox}

{\color{red} All make the sign of the cross with the Officiant as he intones:}

\end{rubricbox}

% TODO: Make sure that the tone of the deus adjutorium matches the season primarily and the solemnity of the feast secondarily
\gresetinitiallines{1}
\gregorioscore{../common/deus-in-adjutorium-lent}

\textit{
O God, come to my assistance.
{\color{red}\Vbar.}~O Lord, make haste to help me.
Glory be to the Father, and to the Son, and to the Holy Spirit,
as it was in the beginning, is now, and ever shall be, world without end. Amen.
Praise to Thee, O Lord, King of endless glory.}

%\vfill\pagebreak}

\section*{Psalm 115}

\textit{\textnormal{Ant. 1.} I will get me to the mountain of myrrh, * and to the hill of frankincense.
 \textnormal{Ps.} I have believed, therefore have I spoken; * but I have been humbled exceedingly.}

\gresetinitiallines{1}
\gregorioscore{ps115-antiphon}

\begin{rubricbox}

{\color{red} The incipit of each Psalm is only used on the first verse.}

\end{rubricbox}

\gresetinitiallines{0}
\gregorioscore{ps115-intonation}

 \begin{latinenglishsection}

\latinenglish{

	\input{../psalms/ps115-1a3} %% 

}{
	%1. I have believed, therefore have I spoken; * but I have been humbled exceedingly.

2. I said in my excess: * Every man is a liar.

3. What shall I render to the Lord, * for all the things that he hath rendered to me?

4. I will take the chalice of salvation; * and I will call upon the name of the Lord.

5.  I will pay my vows to the Lord before all his people: * precious in the sight of the Lord is the death of his saints.

6.  O Lord, for I am thy servant: * I am thy servant, and the son of thy handmaid.

7. Thou hast broken my bonds: * I will sacrifice to thee the sacrifice of praise, and I will call upon the name of the Lord.

8.  I will pay my vows to the Lord in the sight of all his people: * in the courts of the house of the Lord, in the midst of thee, O Jerusalem.

9. Glory be to the Father and to the Son * and to the Holy Spirit.

10. As it was in the beginning, is now, * and ever shall be, world without end. Amen. %%
}

\end{latinenglishsection}

\gresetinitiallines{1}
\gregorioscore{ps115-antiphon}

%%

\section*{Psalm 119}

\textit{\textnormal{Ant. 2.} My beloved * is white and ruddy; the hair of his head is like kingly purple, bound in tresses.
 \textnormal{Ps.} In my trouble I cried to the Lord: * and he heard me.}

\gresetinitiallines{1}
\gregorioscore{ps119-antiphon}

\gresetinitiallines{0}
\gregorioscore{ps119-intonation}

 \begin{latinenglishsection}

\latinenglish{

	2. Dómine, líbera ánimam meam a lábiis in\textbf{í}quis,~* et a lingua \textit{do}\textbf{ló}sa.

3. Quid detur tibi, aut quid apponátur \textbf{ti}bi~* ad linguam \textit{do}\textbf{ló}sam?

4. Sagíttæ poténtis a\textbf{cú}tæ,~* cum carbónibus deso\textit{la}\textbf{tó}riis.

5. Heu mihi! quia incolátus meus prolongátus est:~{\color{red}\GreDagger} habitávi cum habitántibus \textbf{Ce}dar:~* multum íncola fuit áni\textit{ma} \textbf{me}a.

6. Cum his, qui odérunt pacem, eram pa\textbf{cí}ficus:~* cum loquébar illis, impugnábant \textit{me} \textbf{gra}tis.

7. {\color{red}\textit{(bow)}} Glória Patri, et \textbf{Fí}lio,~* et Spirítu\textit{i} \textbf{Sanc}to.

8. {\color{red}\textit{(rise)}} Sicut erat in princípio, et nunc, et \textbf{sem}per,~* et in s\'{\ae}cula sæculó\textit{rum}. \textbf{A}men.
 %%

}{
	1. In my trouble I cried to the Lord: * and he heard me.

2. O Lord, deliver my soul from wicked lips, * and a deceitful tongue.

3. What shall be given to thee, or what shall be added to thee, * to a deceitful tongue?

4. The sharp arrows of the mighty, * with coals that lay waste.

5. Woe is me, that my sojourning is prolonged! I have dwelt with the inhabitants of Cedar: * my soul hath been long a sojourner.

6. With them that hated peace I was peaceable: * when I spoke to them they fought against me without cause.

7. Glory be to the Father, and to the Son, * and to the Holy Ghost.

8. As it was in the beginning, is now, * and ever shall be, world without end. Amen.
 %%
}

\end{latinenglishsection}

\gresetinitiallines{1}
\gregorioscore{ps119-antiphon}

%%

\section*{Psalm 139}

\textit{\textnormal{Ant. 3.} Whither is thy beloved gone, * O thou fairest among women? Whither is thy beloved turned aside?
 \textnormal{Ps.} Deliver me, O Lord, from the evil man: * rescue me from the unjust man.}

\gresetinitiallines{1}
\gregorioscore{ps139-antiphon}

\gresetinitiallines{0}
\gregorioscore{ps139-intonation}

 \begin{latinenglishsection}

\latinenglish{

	\input{../psalms/ps139-3a2} %%

}{
	\input{../psalms/english/ps139} %%
}

\end{latinenglishsection}

\vfill\pagebreak

\gresetinitiallines{1}
\gregorioscore{ps139-antiphon}

%%

\section*{Psalm 140}

\textit{\textnormal{Ant. 4.} A bundle of myrrh is my well-beloved unto me; he shall lie betwixt my breasts.
 \textnormal{Ps.} I have cried to thee, O Lord, hear me: * hearken to my voice, when I cry to thee.}

\gresetinitiallines{1}
\gregorioscore{ps140-antiphon}

\gresetinitiallines{0}
\gregorioscore{ps140-intonation}

 \begin{latinenglishsection}

\latinenglish{

	\input{../psalms/ps140-4g} %%

}{
	2. Let my prayer be directed as incense in thy sight; the lifting up of my hands, as evening sacrifice.
	
3. Set a watch, O Lord, before my mouth: and a door round about my lips.

4. Incline not my heart to evil words; to make excuses in sins. 

5. With men that work iniquity: and I will not communicate with the choicest of them.

6. The just shall correct me in mercy, and shall reprove me: but let not the oil of the sinner fatten my head. 

7. For my prayer also shall still be against the things with which they are well pleased: Their judges falling upon the rock have been swallowed up. 

8. They shall hear my words, for they have prevailed: As when the thickness of the earth is broken up upon the ground: 

9. Our bones are scattered by the side of Hell. But o to thee, O Lord, Lord, are my eyes: in thee have I put my trust, take not away my soul.

10. Keep me from the snare, which they have laid for me, and from the stumbling blocks of them that work iniquity.

11. The wicked shall fall in his net: I am alone until I pass.

12. Glory be to the Father, and to the Son, and to the Holy Spirit.
 	
13. As it was in the beginning, is now, and ever shall be, world without end. Amen. %%
}

\end{latinenglishsection}

\gresetinitiallines{1}
\gregorioscore{ps140-antiphon}

%%

\section*{Psalm 141}

\textit{\textnormal{Ant. 5.} Fair and beautiful art thou, O daughter of Jersualem, terrible as an army in battle array.
 \textnormal{Ps.} Praise the Lord, O Jerusalem: praise thy God, O Sion.}

\gresetinitiallines{1}
\gregorioscore{ps141-antiphon}

\gresetinitiallines{0}
\gregorioscore{ps141-intonation}

 \begin{latinenglishsection}

\latinenglish{

	\input{../psalms/ps141-1f} %%

}{
	2. In his sight I pour out my prayer, and before him I declare my trouble:
	
3. When my spirit failed me, then thou newest my paths. 

4. In this way wherein I walked, they have hidden a snare for me.

5. I looked on my right hand, and beheld, and there was no one that would know me. 

6. Flight hath failed me: and there is no one that hath regard to my soul.

7. I cried to thee, O Lord: I said: Thou art my hope, my portion in the land of the living.

8. Attend to my supplication: for I am brought very low. 

9. Deliver me from my persecutors; for they are stronger than I.

10. Bring my soul out of prison, that I may praise thy name: the just wait for me, until thou reward me.

11. Glory be to the Father, and to the Son, and to the Holy Spirit.
 	
12. As it was in the beginning, is now, and ever shall be, world without end. Amen. %%
}

\end{latinenglishsection}

\gresetinitiallines{1}
\gregorioscore{ps141-antiphon}

%%

\section*{Little Chapter (Isaiah 53:1-2)}

\textit{\color{red}The Officiant leads the Little Chapter:}

\begin{latinenglishsection}

\latinenglish{
	Quis credidit auditui \textbf{nos}tro? et brachium Domini cui reve\textbf{la}tum est? Et ascendet sicut virgul\textbf{tum} \textit{coram} \textbf{eo}, et sicut radix de terra \textbf{si}tienti.
	{\color{red}\Rbar.}~Deo grátias.
}{
	Who hath believed our report? and to whom is the arm of the Lord revealed? And he shall grow up as a tender plant before him, and as a root out of a thirsty ground.
	 {\color{red}\Rbar.}~Thanks be to God.
}

\end{latinenglishsection}

% TODO: Verify that the hymn is correct for the feast (including the responsory after the hymn)

\vfill\pagebreak

\section*{Hymn}

\textit{\color{red}The Cantor leads the hymn:}

\gresetinitiallines{1}
\gregorioscore{../hymns/stabat-mater-dolorosa}

{\itshape

	1.  At the cross her station keeping,
	Stood the mournful Mother weeping,
	Close to Jesus to the last:
	
	2. Through her heart, his sorrow sharing,
	All his bitter anguish bearing,
	Now at length the sword had passed.
	
	3. Oh, how sad and sore distressed
	Was that Mother highly blest
	Of the sole-begotten One!
	
	4. Christ above in torment hangs;
	She beneath beholds the pangs
	Of her dying glorious Son.
	
	5. Is there one who would not weep,
	Whelmed in miseries so deep
	Christ's dear Mother to behold?
	
	6. Can the human heart refrain
	From partaking in her pain,
	In that Mother's pain untold?
	
	7. Bruised, derided, cursed, defiled,
	She beheld her tender Child
	All with bloody scourges rent;
	
	8. For the sins of his own nation,
	Saw him hang in desolation,
	Till his Spirit forth he sent.
	
	9. O thou Mother! fount of love!
	Touch my spirit from above,
	Make my heart with thine accord:
	
	10. Make me feel as thou hast felt;
	Make my soul to glow and melt
	With the love of Christ my Lord.
	Amen.
}

\textit{\color{red}The Cantor says the following before all reply afterwards:}

\gresetinitiallines{0}
\gabcsnippet{
(c3) <c><sp>V/</sp>.</c> O(h)ra(h) pro(h) no(h)bis(h) Vir(h)go(h) do(h)lo(h)ro(h)sis(h)si(h)ma.(g_hVGF'Efgf.) (::)
}

\gresetinitiallines{0}
\gabcsnippet{
(c3) <c><sp>R/</sp>.</c>Ut(h) dig(h)ni(h) ef(h)fi(h)ci(h)a(h)mur(h) pro(h)mis(h)si(h)o(h)ni(h)bus(h) Chris(h)ti.(g_hVGF'Efgf.) (::)
}

\textit{{\color{red}\Vbar.}~Pray for us, O Virgin most sorrowful.
{\color{red}\Rbar.}~That we may be made worthy of the promises of Christ.}

%TODO: Verify that the magnificat antiphon is correct and match the mangificat intonation and pointed text with the tone

\section*{Magnificat}

\textit{\textnormal{Ant Magn.} When Jesus saw His Mother, * and the disciple whom He loved, standing by the Cross, He saith unto His Mother: Woman, behold thy Son! Then saith He to the disciple: Behold thy Mother!
\textnormal{Cant.} My soul doth magnify the Lord: and my spirit hath rejoiced in God my Saviour.}

\begin{rubricbox}

{\color{red}The Cantor leads by intoning the antiphon and the first verse.}

\end{rubricbox}

\gresetinitiallines{1}
\gregorioscore{magnificat-antiphon-only}

\begin{rubricbox}

{\color{red}All \textbf{remain standing} and make the sign of the cross with the Cantor. The incipit of the Magnificat is used at the beginning of every verse:}

\end{rubricbox}

\gresetinitiallines{0}
\gregorioscore{magnificat-intonation}

 \begin{latinenglishsection}

\latinenglish{	
	3. Quia respéxit humilitátem ancíllæ \textbf{su}æ:~* ecce enim ex hoc beátam me dicent omnes gene\textit{ra}\textit{ti}\textbf{ó}nes.

	4. Quia fecit mihi magna qui \textbf{pot}ens est:~* et sanctum \textit{no}\textit{men} \textbf{e}jus.

	5. Et misericórdia ejus a progénie in pro\textbf{gé}nies~* timén\textit{ti}\textit{bus} \textbf{e}um.

	6. Fecit poténtiam in bráchio \textbf{su}o:~* dispérsit supérbos mente \textit{cor}\textit{dis} \textbf{su}i.

	7. Depósuit poténtes de \textbf{se}de,~* et exal\textit{tá}\textit{vit} \textbf{hú}miles.

	8. Esuriéntes implévit \textbf{bo}nis:~* et dívites dimí\textit{sit} \textit{in}\textbf{á}nes.

	9. Suscépit Israël púerum \textbf{su}um,~* recordátus misericór\textit{di}\textit{æ} \textbf{su}æ.

	10. Sicut locútus est ad patres \textbf{nos}tros,~* Abraham et sémini e\textit{jus} \textit{in} \textbf{s\'{\ae}}cula.
}{	
	\input{../psalms/english/magnificat}
}

\end{latinenglishsection}

\textit{\color{red}(bow)} Glória Patri, et \textbf{Fí}lio,~* et Spirí\textit{tu}\textit{i} \textbf{Sanc}to.

\textit{\color{red}(rise)} Sicut erat in princípio, et nunc, et \textbf{sem}per,~* et in s\'{\ae}cula sæcu\textit{ló}\textit{rum}. \textbf{A}men.

\gresetinitiallines{1}
\gregorioscore{magnificat-antiphon-only}

\vfill\pagebreak

%TODO: Verify (with the antiphonary) that the collect is proper for the season. If it is not in antiphonary, use the missal for the feast.

\section*{Collect}

\textit{\color{red}The Officiant leads the collect:}

\begin{latinenglishsection}

\latinenglish{
	{\color{red}\Vbar.}~Dóminus vobíscum.\\
	{\color{red}\Rbar.}~Et cum spíritu túo.
}{
	{\color{red}\Vbar.} The Lord be with you.
	{\color{red}\Rbar.}~And with thy spirit.
}

\textit{\color{red}If the Officiant is not a priest:}

\latinenglish{
	{\color{red}\Vbar.}~Dómine, exáudi oratiónem meam.\\
	{\color{red}\Rbar.}~Et clamor meus ad te véniat.
	
	Orémus.
	
	Deus, in cujus passione, secundum Simeonis prophetiam, dulcissimam animam gloriosæ Virginis et Matris Mariæ doloris gladius\\ pertran\textbf{si}vit: {\color{red}\GreDagger} concede, propitius; ut qui transfixionem ejus et passionem veneran\textbf{do} \textit{reco}\textbf{li}mus, gloriosis meritis et precibus omnium Sanctorum cruci fideliter astantium\\ intercedentibus, passionis tuæ effectum felicem consequamur. {\color{red}*} Qui vivis et regnas cum Deo Patre, in unitáte Spíritus Sancti, Deus, per ómnia s\'{\ae}cula sæculórum.
	
	{\color{red}\Rbar.}~Amen.
}{
	{\color{red}\Vbar.} Lord, hear my prayer.
	{\color{red}\Rbar.}~And let my cry come unto Thee.
	
	Let us pray.
	O God, at Whose Passion, according to the prophecy of Simeon, a sword of sorrow pierced through the gentle soul of the glorious Maiden and Mother Mary, mercifully grant to as many as do ever remember with awe how that her soul was pierced and Thou didst suffer, even for all such be Thou entreated, for the sake and by the prayers of all thy glorious and holy servants who stood so loyally by thy Cross, and grant unto the same, that for them thy life-giving Death may not have been in vain.
	
	Who livest and reignest with God the Father in the unity of the Holy Spirit, God, world without end.
	
	{\color{red}\Rbar.}~Amen.
}
\end{latinenglishsection}

\vfill\pagebreak

% Commemorations

\textit{\color{red}For commemorations, the Cantor intones the antiphon and says the responsorial prayer afterwards. The Officiant prays the associated collect.}

\section*{Commemoration of Friday of Passion Week}

\textit{\textnormal{Ant.} The chief Priests consulted that they might kill Jesus, but they said Not on the Feast-day, lest there be an uproar among the people.}

\gresetinitiallines{1}
\gregorioscore{commemoration-antiphon}

\begin{latinenglishsection}

\latinenglish{
	{\color{red}\Vbar.}.~Eripe me de inimicis meis Deus \textbf{me}us.\\
	{\color{red}\Rbar.}.~Et ab insurgentibus in me libe\textbf{ra} me.
	
	Orémus.
	
	Concede, qu\'{\ae}sumus, omnipotens \textbf{De}us: ut qui protectionis tuæ gratiam quærimus, liberati a \textbf{ma}\textit{lis om}\textbf{ni}bus, secura tibi mente serviamus.{\color{red}*}\\
	Per Dóminum nostrum Jesum Christum, Fílium tuum: qui tecum vivit et regnat in unitáte Spíritus Sancti, Deus, per ómnia s\'{\ae}cula sæculórum.
	
	{\color{red}\Rbar.}~Amen.
}{
	{\color{red}\Vbar.}.~Deliver me from mine enemies, O my God.
	{\color{red}\Rbar.}.~And defend me from them that rise up against me.
	
	Let us pray.
	Grant, we beseech thee, O Almighty God, that we who seek the grace of thy protection, being delivered from all evils, may serve thee ever in peace and quietness of spirit.
	Through Jesus Christ, thy Son our Lord, Who liveth and reigneth with thee, in the unity of the Holy Ghost, God, world without end.
	
	{\color{red}\Rbar.}~Amen.
}

\end{latinenglishsection}

\textit{\color{red}The Officiant leads the following:}

\begin{latinenglishsection}

\latinenglish{
	{\color{red}\Vbar.}~Dóminus vobíscum.\\
	{\color{red}\Rbar.}~Et cum spíritu túo.
}{
	{\color{red}\Vbar.}~The Lord be with you.
	{\color{red}\Rbar.}~And with thy spirit.
}

\textit{\color{red}If the Officiant is not a priest:}

\latinenglish{
	{\color{red}\Vbar.}~Dómine, exáudi oratiónem meam.\\
	{\color{red}\Rbar.}~Et clamor meus ad te véniat.
}{
	{\color{red}\Vbar.}~Lord, hear my prayer. 
	{\color{red}\Rbar.}~And let my cry come unto Thee.
}

\end{latinenglishsection}

%\vfill\pagebreak

\textit{\color{red}The Cantor leads the Benedicamus:}

\gresetinitiallines{1}
\gregorioscore{../common/benedicamus-advent-lent}

{\color{red}\Vbar.}~Let us bless the Lord. {\color{red}\Rbar.}~Thanks be to God.

\vfill\pagebreak

\textit{\color{red}The Officiant leads the following:}

\begin{latinenglishsection}

\latinenglish{
	{\color{red}\Vbar.} Fidélium ánimæ, per misericórdiam\\ Dei, requiéscant in pace. \\
	{\color{red}\Rbar.} Amen.
}{
	May the souls of the faithful departed, through the mercy of God, rest in peace. {\color{red}\Rbar.}~Amen.
}

\end{latinenglishsection}

\section*{Marian Anthem}

\textit{\color{red}The Marian anthem is said kneeling, since it is a weekday. It and the responses afterwards are led by the Cantor; the Officiant leads the ending collect:}

\gresetinitiallines{1}
\gregorioscore{../marian-anthems/ave-regina-caelorum-solemn}

{\itshape
	Hail, O Queen of Heaven.
	Hail, O Lady of Angels
	Hail! thou root, hail! thou gate
	From whom unto the world a light has arisen:
	
	Rejoice, O glorious Virgin,
	Lovely beyond all others,
	Farewell, most beautiful maiden,
	And pray for us to Christ.
}

\begin{latinenglishsection}

\latinenglish{
	{\color{red}\Vbar.} Dignare me laudare te, Virgo sa\textbf{cra}ta.\\
	{\color{red}\Rbar.} Da mihi virtutem contra hostes \textbf{tu}os.
	
	Orémus.
	
	Concede, misericors Deus, fragilitati nostrae praesidium:{\color{red}\GreDagger} ut, qui sanctae Dei Genetricis me\textbf{mo}\textit{riam} \textbf{a}gimus; intercessionis eius auxilio, a nostris iniquitatibus resurgamus.{\color{red}*} 
	Per\\ Christum Dominum nostrum.

	{\color{red}\Rbar.}~Amen.
}{
	{\color{red}\Vbar.}~Vouchsafe that I may praise thee, O sacred Virgin.
	{\color{red}\Rbar.}~Give me strength against thine enemies.
	
	Let us pray.
	We beseech thee, O Lord, mercifully to assist our infirmity: that like as we do now commemorate Blessed Mary Ever-Virgin, Mother of God; so by the help of her intercession we may die to our former sins and rise again to newness of life. Through Christ Our Lord.
	{\color{red}\Rbar.}~Amen.
}
\end{latinenglishsection}

\textit{\color{red}The Officiant says the following in a low recto tono:}

\begin{latinenglishsection}

\latinenglish{
	{\color{red}\Vbar.} Divínum auxílium máneat semper nobíscum.\\
	{\color{red}\Rbar.} Amen.
}{
	{\color{red}\Vbar.}~May the divine assistance remain always with us.
	{\color{red}\Rbar.}~Amen.
}

\end{latinenglishsection}

\begin{rubricbox}

{\color{red}After the Office, all \textbf{kneel} and pray in silence for a time.}

\end{rubricbox}

\end{document}