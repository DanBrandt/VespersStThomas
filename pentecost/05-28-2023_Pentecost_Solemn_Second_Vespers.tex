\documentclass{../vespers-sheet}
\usepackage{multicol}

\begin{document}

\chapter*{Solemn Second Vespers of Pentecost}

\section*{Arrangements}

\begin{rubricbox}

{\color{red}The Second Vespers on Pentecost Sunday is of double rite. The Celebrant must be a priest. Six candles are lit on the altar, as at High Mass, a lectern is placed in front of the sedillia with a red-colored draper over it. The Celebrant is vested in red surplice and cope. If Benediction is to follow, he should be also vested in a stole. There must be a minimum of four ministers/servers to assist in the sanctuary: MC, Thurifer, and 2 Acolytes. Additionally clergy vested in cope (not to exceed six coped ministers) may also be in the sanctuary. All of the assistants and additional clergy wear choir dress.}

\end{rubricbox}

\section*{Beginning of the Office}

\begin{rubricbox}

{\color{red}When the Celebrant kneels, all \textbf{kneel} and pray silently.
Then, when the Celebrant stands, all \textbf{stand} and say silently one \textit{Pater noster} (Our Father) and \textit{Ave Maria} (Hail Mary).
Then all make the sign of the cross with the Celebrant as he intones:}

\end{rubricbox}

 \gresetinitiallines{1}
\gregorioscore{../common/deus-in-adjutorium-solemn}

\textit{
O God, come to my assistance.
{\color{red}\Vbar.}~O Lord, make haste to help me.
Glory be to the Father, and to the Son, and to the Holy Spirit,
as it was in the beginning, is now, and ever shall be, world without end. Amen.
Alleluia.}

\vfill
\pagebreak

\section*{Psalm 109}

\textit{\textnormal{Ant. 1.} When the days of Pentecost were drawing to a close, they were all saying together, alleluia.
 
 \begin{rubricbox}

{\color{red}All remain standing throughout the first antiphon.
After the psalm is intoned by the Cantor, all \textbf{sit} at the asterisk.}

\end{rubricbox}

\gresetinitiallines{1}
\gregorioscore{ps109-intonation}

 \begin{latinenglishsection}

\latinenglish{

	\input{../psalms/ps109-3}

}{
	\input{../psalms/english/ps109}
}

\end{latinenglishsection}

\gresetinitiallines{0}
\gregorioscore{ps109-antiphon}

\vfill\pagebreak

\section*{Psalm 110}

\textit{\textnormal{Ant. 2.} The Spirit of the Lord has filled the whole world, alleluia.
 
 \begin{rubricbox}

{\color{red}The Cantor stands and intones the antiphon, then at the asterisk \textbf{sits} and joins the choir in singing.
The same postures are followed for each of the following Psalms.}

\end{rubricbox}

\gresetinitiallines{1}
\gregorioscore{ps110-antiphon}

 \begin{latinenglishsection}

\latinenglish{
	
	\input{../psalms/ps110-8}

}{
 	\input{../psalms/english/ps110}
}

\end{latinenglishsection}

\gresetinitiallines{0}
\gregorioscore{ps110-antiphon}

\vfill\pagebreak

\section*{Psalm 111}

\textit{\textnormal{Ant. 3.} They were all filled with the Holy Spirit, and began to speak, alleluia.

\gresetinitiallines{1}
\gregorioscore{ps111-intonation}

 \begin{latinenglishsection}

\latinenglish{

	\input{../psalms/ps111-8}
	
}{
	\input{../psalms/english/ps111}
}

\end{latinenglishsection}

\gresetinitiallines{0}
\gregorioscore{ps111-antiphon}

\vfill\pagebreak

\section*{Psalm 112}

\textit{\textnormal{Ant. 4.} Ye fountains, and all things that move in the waters, sing a hymn to God, alleluia.

\gresetinitiallines{1}
\gregorioscore{ps112-intonation}

 \begin{latinenglishsection}

\latinenglish{

	\input{../psalms/ps112-1}
	
}{
	\input{../psalms/english/ps112}
}

\end{latinenglishsection}

\gresetinitiallines{0}
\gregorioscore{ps112-antiphon}

\vfill\pagebreak

\section*{Psalm 113}

\textit{\textnormal{Ant. 5.} The Apostles were speaking with diverse tongues the wonderful works of God, alleluia, alleluia, alleluia.

\gresetinitiallines{1}
\gregorioscore{ps113-intonation}

 \begin{latinenglishsection}

\latinenglish{

	\input{../psalms/ps113-7c2}
	
}{
	\input{../psalms/english/ps113}
}

\end{latinenglishsection}

\gresetinitiallines{1}
\gregorioscore{ps113-antiphon}

\section*{Little Chapter (Acts 2:1-2)}

\textit{\color{red}The Celebrant leads the Little Chapter:}

\begin{latinenglishsection}

\latinenglish{
	 Cum compleréntur dies Pentecóstes, erant omnes discípuli pariter in eódem loco:~\GreDagger\
	 et factus est repénte de c{\oe}lo sonus, tamquam adveniéntis spíritus veheméntis:~*
	 et replévit totam domum, ubi erant sedentes.
	\Rbar.~Deo grátias.
}{
 	When the days of the Pentecost were accomplished, they were all together in one place:
 	and suddenly there came a sound from heaven, as of a mighty wind coming,
 	and it filled the whole house where they were sitting. 
	 \Rbar.~Thanks be to God.
}

\end{latinenglishsection}

\vfill\pagebreak

\section*{Hymn}

\textit{\color{red}The Cantor leads the hymn:}

\gresetinitiallines{1}
\gregorioscore{../hymns/veni-creator-spiritus}

{\itshape

	1. Come, Holy Spirit, Creator blest,
	and in our souls take up Thy rest;
	come with Thy grace and heavenly aid
	to fill the hearts which Thou hast made.
	
	2. O comforter, to Thee we cry,
	O heavenly gift of God Most High,
	O fount of life and fire of love,
	and sweet anointing from above.
	
	3. Thou in Thy sevenfold gifts are known;
	Thou, finger of God's hand we own;
	Thou, promise of the Father, Thou
	Who dost the tongue with power imbue. 
	
	4. Kindle our sense from above,
	and make our hearts o'erflow with love;
	with patience firm and virtue high
	the weakness of our flesh supply.
	
	5. Far from us drive the foe we dread,
	and grant us Thy peace instead;
	so shall we not, with Thee for guide,
	turn from the path of life aside.
	
	6. Oh, may Thy grace on us bestow
	the Father and the Son to know;
	and Thee, through endless times confessed,
	of both the eternal Spirit blest.
	
	7. Now to the Father and the Son,
	Who rose from death, be glory given,
	with Thou, O Holy Comforter,
	henceforth by all in earth and heaven.
	Amen. 
}

\textit{\color{red}The Cantor says the following before all reply afterwards:}

\gresetinitiallines{0}
\gabcsnippet{
(c3) <c><sp>V/</sp>.</c> Lo(h)que(h)bán(h)tur(h) vá(h)ri(h)is(h) lin(h)guis(h) A(h)pó(hi)sto(h)li,(h'_) (,)
al(h)le(fe)lú(f_h)ia.(hiH'Ghih.ghG'FE'fggf.) (::)
}

\gresetinitiallines{0}
\gabcsnippet{
(c3) <c><sp>R/</sp>.</c> Ma(h)gná(h)li(h)a(h) De(hi)i(h'_) (,) al(h)le(fe)lú(f_h)ia.(hiH'Ghih.ghG'FE'fggf.)(::)
}

\textit{{\color{red}\Vbar.}~The Apostles spoke in diverse tongues, alleluia.
{\color{red}\Rbar.}~The wonderful works of God, alleluia.}

\vfill\pagebreak

\section*{Magnificat}

\textit{\textnormal{Ant Magn.} This day * the day of Pentecost is fully come, alleluia. This day the Holy Ghost appeared in fire unto the disciples, and gave unto them gifts of grace. He sent them into all the world, to preach and to testify that he who believeth, and is baptized, shall be saved, alleluia.
}

\begin{rubricbox}

{\color{red}After the Celebrant intones up to the asterisk, all \textbf{sit} and join in singing.}

\end{rubricbox}

\gresetinitiallines{1}
\gregorioscore{magnificat-antiphon-only-solemn}

\begin{rubricbox}

{\color{red}All \textbf{stand} and make the sign of the cross with the cantor.}

\end{rubricbox}

\gresetinitiallines{0}
\gregorioscore{magnificat-intonation}

 \begin{latinenglishsection}

\latinenglish{	
	\input{../psalms/magnificat-1}

}{	
	\input{../psalms/english/magnificat}
}

\end{latinenglishsection}

\begin{rubricbox}

{\color{red}All \textbf{sit} and repeat the antiphon:}

\end{rubricbox}

\gresetinitiallines{1}
\gregorioscore{magnificat-antiphon-only-solemn}
\textit{\textnormal{Ant Magn.} This day * the day of Pentecost is fully come, alleluia. This day the Holy Ghost appeared in fire unto the disciples, and gave unto them gifts of grace. He sent them into all the world, to preach and to testify that he who believeth, and is baptized, shall be saved, alleluia.
}

\section*{Collect}

\begin{rubricbox}

{\color{red}All \textbf{stand} as the celebrant leads the collect:}

\end{rubricbox}

\begin{latinenglishsection}

\latinenglish{
	{\color{red}\Vbar.}~Dóminus vobiscum\\
	{\color{red}\Rbar.}~Et cum spiritu tuo.
	
	Orémus.
	Deus, qui hodiérna die corda fidélium Sancti Spíritus illustratióne docuísti: da nobis in eódem Spíritu recta sápere; et de ejus semper consolatióne gaudére.
	Per Dóminum nostrum Jesum Christum, Fílium tuum: qui tecum vivit et regnat in unitáte Spíritus Sancti, Deus, per ómnia s\'{\ae}cula sæculórum.
	{\color{red}\Rbar.}~Amen. 
}{
	{\color{red}\Vbar.}~The Lord be with you.
	{\color{red}\Rbar.}~And with thy spirit.
	
	Let us pray.
	O God, Who on this day didst teach the hearts of thy faithful people, by the sending to them the light of thine Holy Spirit, grant us by the same Spirit to have a right judgment in all things, and evermore to rejoice in His holy comfort.
	Through Jesus Christ, thy Son our Lord, Who liveth and reigneth with thee, in the unity of the Holy Ghost, God, world without end.
	{\color{red}\Rbar.}~Amen. 
}

\textit{\color{red}The Celebrant leads the following:}

\latinenglish{
	\color{red}\Vbar.}~Dóminus vobiscum\\
	{\color{red}\Rbar.}~Et cum spiritu tuo.
}{
	{\color{red}\Vbar.}~The Lord be with you. {\color{red}\Rbar.}~And with thy spirit.
	{\color{red}\Vbar.}~Let us bless the Lord. {\color{red}\Rbar.}~Thanks be to God.
}

\end{latinenglishsection}

\vfill\pagebreak

\textit{\color{red}The Cantor leads the Benedicamus:}

\gresetinitiallines{1}
\gregorioscore{../common/benedicamus-2v-solem}

\textit{\color{red}The Celebrant leads the following:}

\begin{latinenglishsection}

\latinenglish{
	{\color{red}\Vbar.} Fidélium ánimæ, per misericórdiam Dei, requiéscant in pace. \\
	{\color{red}\Rbar.} Amen.
}{
	May the souls of the faithful departed, through the mercy of God, rest in peace. {\color{red}\Rbar.}~Amen.
}

\latinenglish{
	Pater noster \textit{(silently)}.
}{
	Our Father\dots
}

\latinenglish{
	{\color{red}\Vbar.} Dóminus det nobis suam pacem. \\
	{\color{red}\Rbar.} Et vitam ætérnam. Amen.
}{
	May the Lord grant us his peace. {\color{red}\Rbar.}~And life eternal. Amen.
}

\end{latinenglishsection}

\begin{rubricbox}

\textit{\color{red}The Cantor leads the Marian anthem and responses afterwards; the Celebrant leads the ending collect:}

\end{rubricbox}

\gresetinitiallines{1}
\gregorioscore{../marian-anthems/regina-caeli-solemn}

\gresetinitiallines{0}
\gabcsnippet{
(c3) <c><sp>V/</sp>.</c> Gau(h)de(h) et(h) lae(h)ta(h)re(h), Vir(h)go(h) Ma(h)ri(h)a(h), al(h)le(h)lu(h)ia(f). (::)
}

\gresetinitiallines{0}
\gabcsnippet{
(c3) <c><sp>R/</sp>.</c> Qui(h)a(h) sur(h)re(h)xit(h) Do(h)mi(h)nus(h) ve(h)re(h), al(h)le(h)lu(h)ia(f). (::)
}

\textit{{\color{red}\Vbar.}~Rejoice and be glad, O Virgin Mary, alleluia. 
{\color{red}\Rbar.}~For the Lord has truly risen, alleluia.}

\begin{latinenglishsection}
\latinenglish{
	Orémus.
	Deus, qui per resurrectiónem Fílii tui, Dómini nostri Jesu Christi, mundum lætificáre dignátus es: præsta, qu\'{\ae}sumus; ut per ejus Genetrícem Vírginem Maríam, perpétuæ capiámus gáudia vitæ. Per eúndem Christum Dóminum nostrum.
	{\color{red}\Rbar.}~Amen. 
}{
	Let us pray.
	O God, who gave joy to the world through the resurrection of Thy Son, our Lord Jesus Christ, grant we beseech Thee, that through the intercession of the Virgin Mary, His Mother, we may obtain the joys of everlasting life. Through the same Christ our Lord. 
	{\color{red}\Rbar.}~Amen. 
}
\end{latinenglishsection}

\textit{\color{red}The Celebrant says the following in a low recto tono:}

\begin{latinenglishsection}

\latinenglish{
	{\color{red}\Vbar.} Divínum auxílium máneat semper nobíscum.\\
	{\color{red}\Rbar.} Amen.
}{
	{\color{red}\Vbar.} May the divine assistance remain always with us.
	{\color{red}\Rbar.}~Amen.
}
\end{latinenglishsection}

\begin{rubricbox}

{\color{red}After the Office, all \textbf{kneel} and pray in silence for a time.}

\end{rubricbox}

\end{document}