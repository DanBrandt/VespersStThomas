\documentclass{../vespers-sheet}
\usepackage{multicol}

\begin{document}

\chapter*{Solemn Second Vespers of Pentecost}

\section*{Arrangements}

\begin{rubricbox}

{\color{red}The Second Vespers on Pentecost Sunday is of double rite. The Celebrant must be a priest. Six candles are lit on the altar, as at High Mass, a lectern is placed in front of the sedillia with a red-colored draper over it. The Celebrant is vested in red surplice and cope. If Benediction is to follow, he should be also vested in a stole. There must be a minimum of four ministers/servers to assist in the sanctuary: MC, Thurifer, and 2 Acolytes. Additionally clergy vested in cope (not to exceed six coped ministers) may also be in the sanctuary. All of the assistants and additional clergy wear choir dress.\\
The Celebrant and ministers process in as at Mass; the Celebrant proceeding to in front of the sedilia behind the lectern with the MC at his right, the two acolytes placing their candles on either side of the sanctuary on opposite corners, and the Thurifer carrying nothing but going to his appointed seat.}

\end{rubricbox}

\section*{Beginning of the Office}

\begin{rubricbox}

{\color{red}When the Celebrant kneels, all \textbf{kneel} and pray silently.
Then, when the Celebrant stands, all \textbf{stand} and say silently one \textit{Pater noster} (Our Father) and \textit{Ave Maria} (Hail Mary).
Then all make the sign of the cross with the Celebrant as he intones:}

\end{rubricbox}

 \gresetinitiallines{1}
\gregorioscore{../common/deus-in-adjutorium-solemn}

\textit{
O God, come to my assistance.
{\color{red}\Vbar.}~O Lord, make haste to help me.
Glory be to the Father, and to the Son, and to the Holy Spirit,
as it was in the beginning, is now, and ever shall be, world without end. Amen.
Alleluia.}

\vfill
\pagebreak

\section*{Psalm 109}

\textit{\textnormal{Ant. 1.} When the days of Pentecost were drawing to a close, they were all saying together, alleluia.}
 
 \begin{rubricbox}

{\color{red}All remain standing. The Cantor hums the notes for the first antiphon. The Celebrant then intones the first antiphon up to the asterisk, after which, all cotninue singing the remainder of the antiphon. The Cantor then intones the Psalm up to the asterisk, after which \textbf{all, except the MC, sit}.} 

\end{rubricbox}

\gresetinitiallines{1}
\gregorioscore{ps109-intonation}

 \begin{latinenglishsection}

\latinenglish{

	2. Donec ponam ini\textbf{mí}cos \textbf{tu}os,~* scabéllum \textbf{pe}dum tu\textbf{ó}rum.

3. Virgam virtútis tuæ emíttet Dómi\textbf{nus} ex \textbf{Si}on:~* domináre in médio inimi\textbf{có}rum tu\textbf{ó}rum.

4. Tecum princípium in die virtútis tuæ in splendóri\textbf{bus} sanc\textbf{tó}rum:~* ex útero ante lucíferum \textbf{gé}nu\textbf{i} te.

5. Jurávit Dóminus, et non pœni\textbf{té}bit \textbf{e}um:~* Tu es sacérdos in ætérnum secúndum órdi\textbf{nem} Mel\textbf{chí}sedech.

6. Dóminus a \textbf{dex}tris \textbf{tu}is,~* confrégit in die iræ \textbf{su}æ \textbf{re}ges.

7. Judicábit in natiónibus, im\textbf{plé}bit ru\textbf{í}nas:~* \\ conquassábit cápita in \textbf{ter}ra mul\textbf{tó}rum.

8. De torrénte in \textbf{vi}a \textbf{bi}bet:~* proptérea exal\textbf{tá}bit \textbf{ca}put.

{\color{red}\textit{(bow)}} Glória \textbf{Pa}tri, et \textbf{Fí}\textbf{li}o,~* et Spi\textbf{rí}tui \textbf{Sanc}to.

{\color{red}\textit{(rise)}} Sicut erat in princípio, et \textbf{nunc}, et \textbf{sem}per,~* et in s\'{\ae}cula sæcu\textbf{ló}rum. \textbf{A}men.

}{
	% 1. The Lord said to my Lord: Sit thou at my right hand:

2. Until I make thy enemies thy footstool.
 
3. The Lord will send forth the sceptre of thy power out of Sion: rule thou in the midst of thy enemies.
 
4. With thee is the principality in the day of thy strength: in the brightness of the saints:
 from the womb before the day star I begot thee.
 
5. The Lord hath sworn, and he will not repent: Thou art a priest for ever according to the order of Melchisedech.
 
6. The Lord at thy right hand hath broken kings in the day of his wrath.

7. He shall judge among nations, he shall fill ruins: he shall crush the heads in the land of the many.

8. He shall drink of the torrent in the way: therefore shall he lift up the head. 

Glory be.
}

\end{latinenglishsection}

 \begin{rubricbox}

{\color{red}Remaining seated, all repaat the antiphon.} 

\end{rubricbox}

\gresetinitiallines{0}
\gregorioscore{ps109-antiphon}

\vfill\pagebreak

\section*{Psalm 110}

\textit{\textnormal{Ant. 2.} The Spirit of the Lord has filled the whole world, alleluia.}
 
 \begin{rubricbox}

{\color{red}The Cantor alone intones the antiphon, then at the asterisk \textbf{sits} and joins the choir in singing.
The same postures are followed for each of the following Psalms.}

\end{rubricbox}

\gresetinitiallines{1}
\gregorioscore{ps110-antiphon}

 \begin{latinenglishsection}

\latinenglish{
	
	2. Magna ópera \textbf{Dó}mini:~*
	exquisíta in omnes volun\textit{tá}\textit{tes} \textbf{e}jus.

3. Conféssio et magnificéntia opus \textbf{e}jus:~*
	et justítia ejus manet in s\'{\ae}\textit{cu}\-\textit{lum} \textbf{s\'{\ae}}culi.

4. Memóriam fecit mirabílium suórum,~{\color{red}\GreDagger}\
	miséricors et miserátor \textbf{Dó}\-minus:~*
	escam dedit ti\textit{mén}\textit{ti}\textbf{bus} se.

5. Memor erit in s\'{\ae}culum testaménti \textbf{su}i:~*
	virtútem óperum suórum annuntiábit pó\textit{pu}\textit{lo} \textbf{su}o:

6. Ut det illis hereditátem \textbf{gén}ti\-um:~*
	ópera mánuum ejus véritas, \textit{et} \textit{ju}\textbf{dí}\-cium.

7. Fidélia ómnia mandáta ejus:~{\color{red}\GreDagger}\
	confirmáta in s\'{\ae}culum \textbf{s\'{\ae}}culi,~*
	facta in veritáte et \textit{æ}\textit{qui}\textbf{tá}te.

8. Redemptiónem misit pópulo \textbf{su}o:~*
	mandávit in ætérnum\\ testa\textit{mén}\textit{tum} \textbf{su}um.

9. Sanctum, et terríbile nomen \textbf{e}jus:~*
	inítium sapiéntiæ \textit{ti}\textit{mor} \textbf{Dó}\-mini.

10. Intelléctus bonus ómnibus faciéntibus \textbf{e}um:~*
	laudátio ejus manet in s\'{\ae}\textit{cu}\textit{lum} \textbf{s\'{\ae}}culi.

{\color{red}\textit{(bow)}} Glória Patri, et \textbf{Fí}lio,~*
	et Spirí\textit{tu}\textit{i} \textbf{Sanc}to.

{\color{red}\textit{(rise)}} Sicut erat in princípio, et nunc, et \textbf{sem}per,~*
	et in s\'{\ae}cula sæcu\textit{ló}\textit{rum}. \textbf{A}men.

}{
 	 1. I will praise thee, O Lord, with my whole heart; in the council of the just: and in the congregation.
 
 2. Great are the works of the Lord: sought out according to all his wills.
 
 3. His work is praise and magnificence: and his justice continueth for ever and ever.
 
 4.  He hath made a remembrance of his wonderful works, being a merciful and gracious Lord: he hath given food to them that fear him. 	
 
 5. He will be mindful for ever of his covenant: he will shew forth to his people the power of his works.
 
 6. That he may give them the inheritance of the Gentiles: the works of his hands are truth and judgment.
 
 7. All his commandments are faithful: confirmed for ever and ever, made in truth and equity.
 
 8. He hath sent redemption to his people: he hath commanded his covenant for ever.
 
 9. Holy and terrible is his name: the fear of the Lord is the beginning of wisdom.
 
 10. A good understanding to all that do it: his praise continueth for ever and ever. 
}

\end{latinenglishsection}

\gresetinitiallines{0}
\gregorioscore{ps110-antiphon}

\vfill\pagebreak

\section*{Psalm 111}

\textit{\textnormal{Ant. 3.} They were all filled with the Holy Spirit, and began to speak, alleluia.}

\gresetinitiallines{1}
\gregorioscore{ps111-intonation}

 \begin{latinenglishsection}

\latinenglish{

	2. Potens in terra erit semen \textbf{e}jus:~*
	generátio rectórum be\textit{ne}\textit{di}\textbf{cé}tur.

3. Glória, et divítiæ in domo \textbf{e}jus:~*
	et justítia ejus manet in s\'{\ae}\textit{cu}\textit{lum} \textbf{s\'{\ae}}culi.

4. Exórtum est in ténebris lumen \textbf{rec}tis:~*
	miséricors, et miserá\textit{tor}, \textit{et} \textbf{jus}tus.

5. Jucúndus homo qui miserétur et cómmodat,~{\color{red}\GreDagger}\
	dispónet sermónes suos in ju\textbf{dí}cio:~* quia in ætérnum non \textit{com}\textit{mo}\textbf{vé}bitur.

6. In memória ætérna erit \textbf{jus}tus:~*
	ab auditióne mala \textit{non} \textit{ti}\textbf{mé}bit.

7. Parátum cor ejus speráre in Dómino,~{\color{red}\GreDagger}\
	confirmátum est cor \textbf{e}jus:~* non commovébitur donec despíciat ini\textit{mí}\textit{cos} \textbf{su}os.

8. Dispérsit, dedit paupéribus:~{\color{red}\GreDagger}\
	justítia ejus manet in s\'{\ae}culum \textbf{s\'{\ae}}\-culi,~*
	cornu ejus exaltábi\textit{tur} \textit{in} \textbf{gló}ria.

9. Peccátor vidébit, et irascétur,~{\color{red}\GreDagger}\
	déntibus suis fremet et ta\textbf{bé}scet:~*
	desidérium peccató\textit{rum} \textit{per}\textbf{í}bit.

{\color{red}\textit{(bow)}} Glória Patri, et \textbf{Fí}lio,~*
	et Spirí\textit{tu}\textit{i} \textbf{Sanc}to.

{\color{red}\textit{(rise)}} Sicut erat in princípio, et nunc, et \textbf{sem}per,~*
	et in s\'{\ae}cula sæcu\textit{ló}\textit{rum}. \textbf{A}men.
	
}{
	1. Blessed is the man that feareth the Lord: he shall delight exceedingly in his commandments.

2. His seed shall be mighty upon earth: the generation of the righteous shall be blessed.

3. Glory and wealth shall be in his house: and his justice remaineth for ever and ever.

4. To the righteous a light is risen up in darkness: he is merciful, and compassionate and just.

5. Acceptable is the man that sheweth mercy and lendeth: he shall order his words with judgment:
because he shall not be moved for ever.

6. The just shall be in everlasting remembrance: he shall not fear the evil hearing.

7. His heart is ready to hope in the Lord: his heart is strengthened, he shall not be moved until he look over his enemies.

8. He hath distributed, he hath given to the poor: his justice remaineth for ever and ever: his horn shall be exalted in glory.

9. The wicked shall see, and shall be angry, he shall gnash with his teeth and pine away: the desire of the wicked shall perish. 
}

\end{latinenglishsection}

\gresetinitiallines{0}
\gregorioscore{ps111-antiphon}

\vfill\pagebreak

\section*{Psalm 112}

\textit{\textnormal{Ant. 4.} Ye fountains, and all things that move in the waters, sing a hymn to God, alleluia.}

 \begin{rubricbox}

{\color{red}In this Psalm, all proceeds as before, except that \textbf{all} slightly bow at the words `\textit{Sit nomen Domini benedictum}' in the second verse. The Thurifer retrieves the incense and the thurible.}

\end{rubricbox}

\gresetinitiallines{1}
\gregorioscore{ps112-intonation}

 \begin{latinenglishsection}

\latinenglish{

	2. Sit nomen Dómini \textbf{be}ne\textbf{díc}tum,~* ex hoc nunc, et us\textit{que} \textit{in} \textbf{s\'{\ae}}culum.

3. A solis ortu usque \textbf{ad} oc\textbf{cá}sum,~* laudábile \textit{no}\textit{men} \textbf{Dó}mini.

4. Excélsus super omnes \textbf{gen}tes \textbf{Dó}minus,~* et super cælos gló\textit{ri}\textit{a} \textbf{e}jus.

5. Quis sicut Dóminus, Deus noster, qui in \textbf{al}tis \textbf{há}bitat,~* et humília réspicit in cælo \textit{et} \textit{in} \textbf{ter}ra?

6. Súscitans a \textbf{ter}ra \textbf{ín}opem,~* et de stércore é\textit{ri}\textit{gens} \textbf{páu}perem:

7. Ut cóllocet eum \textbf{cum} prin\textbf{cí}pibus,~* cum princípibus pó\textit{pu}\textit{li} \textbf{su}i.

8. Qui habitáre facit stéri\textbf{lem} in \textbf{do}mo,~* matrem filió\textit{rum} \textit{læ}\textbf{tán}tem.

{\color{red}\textit{(bow)}} Glória \textbf{Pa}tri, et \textbf{Fí}lio,~* et Spirí\textit{tu}\textit{i} \textbf{Sanc}to.

{\color{red}\textit{(rise)}} Sicut erat in princípio, et \textbf{nunc}, et \textbf{sem}per,~* et in s\'{\ae}cula sæcu\textit{ló}\textit{rum}. \textbf{A}men.
	
}{
	%1. Praise the Lord, ye children: praise ye the name of the Lord.
 	
2. Blessed be the name of the Lord, from henceforth now and for ever.
 	
3. From the rising of the sun unto the going down of the same, the name of the Lord is worthy of praise.
 	
4. The Lord is high above all nations; and his glory above the heavens.
 	
5.Who is as the Lord our God, who dwelleth on high, and looketh down on the low things in heaven and in earth?
 	
6. Raising up the needy from the earth, and lifting up the poor out of the dunghill:
 	
7. That he may place him with princes, with the princes of his people.
 	
8. Who maketh a barren woman to dwell in a house, the joyful mother of children. 

Glory be.
}

\end{latinenglishsection}

\gresetinitiallines{0}
\gregorioscore{ps112-antiphon}

\vfill\pagebreak

\section*{Psalm 113}

\textit{\textnormal{Ant. 5.} The Apostles were speaking with diverse tongues the wonderful works of God, alleluia, alleluia, alleluia.}

 \begin{rubricbox}

{\color{red}In verse 23, the two acolytes rise, make their reverences, and relight their candles. At the `Sicut era', they pick up their candles, meet and genuflect in the middle, arrive at the lectern, bow to the Celerbant, and stand at opposing sides of the lectern facing each other.}

 \end{rubricbox}

\gresetinitiallines{1}
\gregorioscore{ps113-intonation}

 \begin{latinenglishsection}

\latinenglish{

	2. Facta est Jud\'{\ae}a sanctifi\textbf{cá}tio \textbf{e}jus,~* 
	Israël pot\textbf{és}tas \textbf{e}jus.

3. Mare \textbf{vi}dit, et \textbf{fu}git:~* 
	Jordánis convérsus \textbf{est} re\textbf{trór}sum.

4. Montes exsultavérunt \textbf{ut} a\textbf{rí}\textbf{e}tes,~* 
	et colles sicut \textbf{a}gni \textbf{ó}vium.

5. Quid est tibi, mare, \textbf{quod} fu\textbf{gís}ti:~* 
	et tu, Jordánis, quia convérsus \textbf{es} re\textbf{trór}sum?

6. Montes, exsultástis \textbf{sic}ut a\textbf{rí}\textbf{e}tes,~* 
	et colles, sicut \textbf{a}gni \textbf{ó}vium.

7. A fácie Dómini \textbf{mo}ta est \textbf{ter}ra,~* 
	a fácie \textbf{De}i \textbf{Ja}cob.

8. Qui convértit petram in \textbf{sta}gna a\textbf{quá}rum,~* 
	et rupem in \textbf{fon}tes a\textbf{quá}rum.

9. Non nobis, Dómi\textbf{ne}, non \textbf{no}bis:~* 
	sed nómini \textbf{tu}o da \textbf{gló}riam.

10. Super misericórdia tua, et veri\textbf{tá}te \textbf{tu}a:~* 
	nequándo dicant gentes: Ubi est \textbf{De}us e\textbf{ó}rum?

11. Deus autem \textbf{nos}ter in \textbf{cæ}lo:~* 
	ómnia quæcúmque \textbf{vó}luit, \textbf{fe}cit.

12. Simulácra géntium ar\textbf{gén}tum, et \textbf{au}rum,~* 
	ópera \textbf{má}nuum \textbf{hó}minum.

13. Os habent, et \textbf{non} lo\textbf{quén}tur:~* 
	óculos habent, et \textbf{non} vi\textbf{dé}bunt.

14. Aures habent, \textbf{et} non \textbf{áu}\textbf{di}ent:~* 
	nares habent, et non \textbf{o}do\textbf{rá}bunt.

15. Manus habent, et non palpábunt:~{\color{red}\GreDagger}\ pedes habent, et non \textbf{am}bu\textbf{lá}bunt:~* 
	non clamábunt in \textbf{gút}ture \textbf{su}o.

16. Símiles illis fiant qui \textbf{fá}ciunt \textbf{e}a:~* 
	et omnes qui con\textbf{fí}dunt in \textbf{e}is.

17. Domus Israël spe\textbf{rá}vit in \textbf{Dó}\textbf{mi}no:~* 
	adjútor eórum et pro\textbf{téc}tor e\textbf{ó}rum est,

18. Domus Aaron spe\textbf{rá}vit in \textbf{Dó}\textbf{mi}no:~* 
	adjútor eórum et pro\textbf{téc}tor e\textbf{ó}rum est,

19. Qui timent Dóminum, spera\textbf{vé}runt in \textbf{Dó}\textbf{mi}no:~* 
	adjútor eórum et pro\textbf{téc}tor e\textbf{ó}rum est.

20. Dóminus memor \textbf{fu}it \textbf{nos}tri:~* 
	et bene\textbf{dí}xit \textbf{no}bis:

21. Benedíxit \textbf{dó}mui \textbf{Is}\textbf{ra}ël:~* 
	benedíxit \textbf{dó}mui \textbf{A}aron.

22. Benedíxit ómnibus, qui \textbf{ti}ment \textbf{Dó}\textbf{mi}num,~* 
	pusíllis \textbf{cum} ma\textbf{jó}ribus.

23. Adjíciat \textbf{Dó}minus \textbf{su}\textbf{per} vos:~* 
	super vos, et super \textbf{fí}lios \textbf{ves}tros.

24. Benedícti \textbf{vos} a \textbf{Dó}\textbf{mi}no,~* 
	qui fecit \textbf{cæ}lum, et \textbf{ter}ram.

25. Cælum \textbf{cæ}li \textbf{Dó}\textbf{mi}no:~* 
	terram autem dedit \textbf{fí}liis \textbf{hó}minum.

26. Non mórtui lau\textbf{dá}bunt te, \textbf{Dó}\textbf{mi}ne:~* 
	neque omnes, qui descéndunt \textbf{in} in\textbf{fér}num.

27. Sed nos qui vívimus, bene\textbf{dí}cimus \textbf{Dó}\textbf{mi}no,~* 
	ex hoc nunc et \textbf{us}que in \textbf{s\'{\ae}}culum.

\textit{(bow)} Glória \textbf{Pa}tri, et \textbf{Fí}\textbf{li}o,~* 
	et Spi\textbf{rí}tui \textbf{Sanc}to.

\textit{(rise)} Sicut erat in princípio, et \textbf{nunc}, et \textbf{sem}per,~* 
	et in s\'{\ae}cula sæcu\textbf{ló}rum. \textbf{A}men.

	
}{
	1. When Israel went out of Egypt, the house of Jacob from a barbarous people:

2. Judea was made his sanctuary, Israel his dominion.

3. The sea saw and fled: Jordan was turned back.

4. The mountains skipped like rams, and the hills like the lambs of the flock.

5. What ailed thee, O thou sea, that thou didst flee: and thou, O Jordan, that thou wast turned back?

6. Ye mountains, that ye skipped like rams, and ye hills, like lambs of the flock?

7. At the presence of the Lord the earth was moved, at the presence of the God of Jacob:

8. Who turned the rock into pools of water, and the stony hill into fountains of waters.

9. Not to us, O Lord, not to us; but to thy name give glory.

10. For thy mercy, and for thy truth's sake: lest the gentiles should say: Where is their God?

11. But our God is in heaven: he hath done all things whatsoever he would.

12. The idols of the gentiles are silver and gold, the works of the hands of men.

13. They have mouths and speak not: they have eyes and see not.

14. They have ears and hear not: they have noses and smell not.

15. They have hands and feel not: they have feet and walk not: neither shall they cry out through their throat.

16. Let them that make them become like unto them: and all such as trust in them.

17. The house of Israel hath hoped in the Lord: he is their helper and their protector.

18. The house of Aaron hath hoped in the Lord: he is their helper and their protector.

19. They that fear the Lord hath hoped in the Lord: he is their helper and their protector.

20. The Lord hath been mindful of us, and hath blessed us.

21. He hath blessed the house of Israel: he hath blessed the house of Aaron.

22. He hath blessed all that fear the Lord, both little and great.

23. May the Lord add blessings upon you: upon you, and upon your children.

24. Blessed be you of the Lord, who made heaven and earth.

25. The heaven of heaven is the Lord's: but the earth he has given to the children of men.

26. The dead shall not praise thee, O Lord: nor any of them that go down to hell.

27. But we that live bless the Lord: from this time now and for ever.
}

\end{latinenglishsection}

\gresetinitiallines{0}
\gregorioscore{ps113-antiphon}

\section*{Little Chapter (Acts 2:1-2)}

\textit{\color{red}\textbf{All stand}. The Celebrant leads the Little Chapter:}

\begin{latinenglishsection}

\latinenglish{
	 Cum compleréntur dies Pentecóstes, erant omnes discípuli pariter in eódem loco:~\GreDagger\
	 et factus est repénte de c{\oe}lo sonus, tamquam adveniéntis spíritus veheméntis:~*
	 et replévit totam domum, ubi erant sedentes.
	\Rbar.~Deo grátias.
}{
 	When the days of the Pentecost were accomplished, they were all together in one place:
 	and suddenly there came a sound from heaven, as of a mighty wind coming,
 	and it filled the whole house where they were sitting. 
	 \Rbar.~Thanks be to God.
}

\end{latinenglishsection}

\vfill\pagebreak

\section*{Hymn}

\textit{\color{red}The Celebrant intones the hymn. For this hymn, \textbf{all} kneel for the first verse and then stand for the remaining verses. The two acolytes turn toward the Celebrant, bow, and return their candles to the altar step, where they remain lit.}

\gresetinitiallines{1}
\gregorioscore{../hymns/veni-creator-spiritus_solemn_vespers}

{\itshape

	1. Come, Holy Spirit, Creator blest,
	and in our souls take up Thy rest;
	come with Thy grace and heavenly aid
	to fill the hearts which Thou hast made.
	
	2. O comforter, to Thee we cry,
	O heavenly gift of God Most High,
	O fount of life and fire of love,
	and sweet anointing from above.
	
	3. Thou in Thy sevenfold gifts are known;
	Thou, finger of God's hand we own;
	Thou, promise of the Father, Thou
	Who dost the tongue with power imbue. 
	
	4. Kindle our sense from above,
	and make our hearts o'erflow with love;
	with patience firm and virtue high
	the weakness of our flesh supply.
	
	5. Far from us drive the foe we dread,
	and grant us Thy peace instead;
	so shall we not, with Thee for guide,
	turn from the path of life aside.
	
	6. Oh, may Thy grace on us bestow
	the Father and the Son to know;
	and Thee, through endless times confessed,
	of both the eternal Spirit blest.
	
	7. Now to the Father and the Son,
	Who rose from death, be glory given,
	with Thou, O Holy Comforter,
	henceforth by all in earth and heaven.
	Amen. 
}

\textit{\color{red}The Cantor says the versicle, and all reply:}

\gresetinitiallines{0}
\gabcsnippet{
(c3) <c><sp>V/</sp>.</c> Lo(h)que(h)bán(h)tur(h) vá(h)ri(h)is(h) lin(h)guis(h) A(h)pó(hi)sto(h)li,(h'_) (,)
al(h)le(fe)lú(f_h)ia.(hiH'Ghih.ghG'FE'fggf.) (::)
}

\gresetinitiallines{0}
\gabcsnippet{
(c3) <c><sp>R/</sp>.</c> Ma(h)gná(h)li(h)a(h) De(hi)i(h'_) (,) al(h)le(fe)lú(f_h)ia.(hiH'Ghih.ghG'FE'fggf.)(::)
}

\textit{{\color{red}\Vbar.}~The Apostles spoke in diverse tongues, alleluia.
{\color{red}\Rbar.}~The wonderful works of God, alleluia.}

\section*{Magnificat}

\textit{\textnormal{Ant Magn.} This day * the day of Pentecost is fully come, alleluia. This day the Holy Ghost appeared in fire unto the disciples, and gave unto them gifts of grace. He sent them into all the world, to preach and to testify that he who believeth, and is baptized, shall be saved, alleluia.
}

\begin{rubricbox}

{\color{red}After the Celebrant intones up to the asterisk, all \textbf{sit} and join in singing.}

\end{rubricbox}

\gresetinitiallines{1}
\gregorioscore{magnificat-antiphon}

\begin{rubricbox}

{\color{red}All \textbf{stand}. The MC lifts the Celebrant's cope, and all make the Sign of the Cross with \textbf{the Cantor} as he intones the Magnificat. The Celebrant, MC, and Thurifer incense the altar, as at Mass, the Celebrant saying the Magnificat silently while this is done. Afterwards, they return to the sedilia where the MC incenses the Celebrant. The Thurifer then incenses the MC, the two Acolytes, others in choir, and finally the congregation. He then remains in his place. \textbf{The `Gloria Patri' is not said until the congregation has been incensed}.}

\end{rubricbox}

\vfill\pagebreak

\gresetinitiallines{0}
\gregorioscore{magnificat-intonation}

 \begin{latinenglishsection}

\latinenglish{	
	3. Quia respéxit humilitátem \textit{an}\textit{cíl}\textit{læ} \textbf{su}æ:~* ecce enim ex hoc beátam me dicent omnes gene\textit{ra}\textit{ti}\textbf{ó}nes.

4. Quia fecit mihi \textit{ma}\textit{gna} \textit{qui} \textbf{pot}\textbf{ens} est:~* et sanctum \textit{no}\textit{men} \textbf{e}jus.

5. Et misericórdia ejus a progéni\textit{e} \textit{in} \textit{pro}\textbf{gé}\textbf{ni}es~* timén\textit{ti}\textit{bus} \textbf{e}um.

6. Fecit poténtiam in \textit{brá}\textit{chi}\textit{o} \textbf{su}o:~* dispérsit supérbos mente \textit{cor}\textit{dis} \textbf{su}i.

7. Depósuit pot\textit{én}\textit{tes} \textit{de} \textbf{se}de,~* et exal\textit{tá}\textit{vit} \textbf{hú}miles.

8. Esuriéntes \textit{im}\textit{plé}\textit{vit} \textbf{bo}nis:~* et dívites dimí\textit{sit} \textit{in}\textbf{á}nes.

9. Suscépit Israël \textit{pú}\textit{e}\textit{rum} \textbf{su}um,~* recordátus misericór\textit{di}\textit{æ} \textbf{su}æ.

10. Sicut locútus est \textit{ad} \textit{pa}\textit{tres} \textbf{nos}tros,~* Abraham et sémini e\textit{jus} \textit{in} \textbf{s\'{\ae}}cula.

11. {\color{red}\textit{(bow)}} Glória \textit{Pa}\textit{tri}, \textit{et} \textbf{Fí}\textbf{li}o,~* et Spirí\textit{tu}\textit{i} \textbf{Sanc}to.

12. {\color{red}\textit{(rise)}} Sicut erat in princípio, \textit{et} \textit{nunc}, \textit{et} \textbf{sem}per,~* et in s\'{\ae}cula sæcu\textit{ló}\textit{rum}. \textbf{A}men.

}{	
	1. My soul doth magnify the Lord.

2. And my spirit hath rejoiced in God my Saviour.

3. Because he hath regarded the humility of his handmaid; for behold from henceforth all generations shall call me blessed.

4. Because he that is mighty, hath done great things to me; and holy is his name.

5. And his mercy is from generation unto generations, to them that fear him.

6. He hath shewed might in his arm: he hath scattered the proud in the conceit of their heart.

7. He hath put down the mighty from their seat, and hath exalted the humble.

8. He hath filled the hungry with good things; and the rich he hath sent empty away.

9. He hath received Israel his servant, being mindful of his mercy: 

10. As he spoke to our fathers, to Abraham and to his seed for ever. 
}

\end{latinenglishsection}

\begin{rubricbox}

{\color{red}All sing the Magnifcat antiphon:}

\end{rubricbox}

\gresetinitiallines{1}
\gregorioscore{magnificat-antiphon}

\vfill\pagebreak

\section*{Collect}

\begin{rubricbox}

{\color{red}The two acolytes retrive their candles and go to either side of the Lectern. All \textbf{stand} as the Celebrant leads the collect:}

\end{rubricbox}

\begin{latinenglishsection}

\latinenglish{
	{\color{red}\Vbar.}~Dóminus vobiscum\\
	{\color{red}\Rbar.}~Et cum spiritu tuo.
	
	Orémus.
	Deus, qui hodiérna die corda fidélium Sancti Spíritus illustratióne docuísti: da nobis in eódem Spíritu recta sápere; et de ejus semper consolatióne gaudére.
	Per Dóminum nostrum Jesum Christum, Fílium tuum: qui tecum vivit et regnat in unitáte ejúsdem Spíritus Sancti, Deus, per ómnia s\'{\ae}cula sæculórum.
	{\color{red}\Rbar.}~Amen. 
}{
	{\color{red}\Vbar.}~The Lord be with you.
	{\color{red}\Rbar.}~And with thy spirit.
	
	Let us pray.
	O God, Who on this day didst teach the hearts of thy faithful people, by the sending to them the light of thine Holy Spirit, grant us by the same Spirit to have a right judgment in all things, and evermore to rejoice in His holy comfort.
	Through Jesus Christ, thy Son our Lord\dots\ in the unity of the same Holy Ghost.
	{\color{red}\Rbar.}~Amen. 
}

\latinenglish{
	{\color{red}\Vbar.}~Dóminus vobiscum.\\
	{\color{red}\Rbar.}~Et cum spiritu tuo.
}{
	{\color{red}\Vbar.}~The Lord be with you. {\color{red}\Rbar.}~And with thy spirit.
	{\color{red}\Vbar.}~Let us bless the Lord. {\color{red}\Rbar.}~Thanks be to God.
}

\end{latinenglishsection}

\textit{\color{red}The Cantor leads the Benedicamus, and all respond:}

\gresetinitiallines{1}
\gregorioscore{../common/benedicamus-2v-solem}

\textit{\color{red}The Celebrant leads the following slowly, in a low recto tono:}

\begin{latinenglishsection}

\latinenglish{
	{\color{red}\Vbar.} Fidélium ánimæ, per misericórdiam Dei, requiéscant in pace. \\
	{\color{red}\Rbar.} Amen.
}{
	May the souls of the faithful departed, through the mercy of God, rest in peace. {\color{red}\Rbar.}~Amen.
}

\latinenglish{
	Pater noster \textit{(silently)}.
}{
	Our Father\dots
}

\latinenglish{
	{\color{red}\Vbar.} Dóminus det nobis suam pacem. \\
	{\color{red}\Rbar.} Et vitam ætérnam. Amen.
}{
	May the Lord grant us his peace. {\color{red}\Rbar.}~And life eternal. Amen.
}

\end{latinenglishsection}

\begin{rubricbox}

{\color{red}The Cantor leads the Marian anthem and responses afterwards; the Celebrant leads the ending collect:}

\end{rubricbox}

\gresetinitiallines{1}
\gregorioscore{../marian-anthems/regina-caeli-solemn}

\vfill\pagebreak

\begin{latinenglishsection}
\latinenglish{
	{\color{red}\Vbar.}~Gaude et lætáre, Virgo María, allelúia.\\
	{\color{red}\Rbar.}~Quia surréxit Dóminus vere, allelúia.
	
	Orémus.
	Deus, qui per resurrectiónem Fílii tui, Dómini nostri Jesu Christi, mundum lætificáre dignátus es: præsta, qu\'{\ae}sumus; ut per ejus Genetrícem Vírginem Maríam, perpétuæ capiámus gáudia vitæ. Per eúndem Christum Dóminum nostrum.
	{\color{red}\Rbar.}~Amen. 
}{
	{\color{red}\Vbar.}~Rejoice and be glad, O Virgin Mary, alleluia. 
	{\color{red}\Rbar.}~For the Lord has truly risen, alleluia.

	Let us pray.
	O God, who gave joy to the world through the resurrection of Thy Son, our Lord Jesus Christ, grant we beseech Thee, that through the intercession of the Virgin Mary, His Mother, we may obtain the joys of everlasting life. Through the same Christ our Lord. 
	{\color{red}\Rbar.}~Amen. 
}
\end{latinenglishsection}

\textit{\color{red}The Celebrant says the following slowly, in a low recto tono:}

\begin{latinenglishsection}

\latinenglish{
	{\color{red}\Vbar.} Divínum auxílium máneat semper nobíscum.\\
	{\color{red}\Rbar.} Amen.
}{
	{\color{red}\Vbar.} May the divine assistance remain always with us.
	{\color{red}\Rbar.}~Amen.
}
\end{latinenglishsection}

\begin{rubricbox}

{\color{red}The Office is concluded. The Celebrant and ministers recess, while all else kneel and pray for a time.}

\end{rubricbox}

\end{document}