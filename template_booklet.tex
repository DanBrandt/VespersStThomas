\documentclass{../vespers-booklet}
\usepackage{multicol}

\begin{document}

% TODO: Update the title for the specific feast
\chapter*{Vespers}

%\section*{Beginning of the Office}

\begin{rubricbox}

{\color{red}When the Officiant kneels, all \textbf{kneel} and pray silently.
Then, when the Officiant stands, all \textbf{stand} and say silently one \textit{Pater noster} (Our Father) and \textit{Ave Maria} (Hail Mary).
Then all make the sign of the cross with the Officiant as he intones:}

\end{rubricbox}

% TODO: Make sure that the tone of the deus adjutorium matches the season primarily and the solemnity of the feast secondarily
 \gresetinitiallines{1}
\gregorioscore{../common/deus-in-adjutorium-lent}

\textit{
O God, come to my assistance.
\Vbar.~O Lord, make haste to help me.
Glory be to the Father, and to the Son, and to the Holy Spirit,
as it was in the beginning, is now, and ever shall be, world without end. Amen.
Praise to Thee, O Lord, King of endless glory.}

\vfill\pagebreak

%TODO: Add that the correct psalms, and verify that their tones, and their associated pointed text are correct

\section*{Psalm XXX}

\textit{\textnormal{Ant. 1.} .
 \textnormal{Ps.} .}
 
 \begin{rubricbox}

{\color{red}All remain standing throughout the first antiphon.
After the psalm is intoned by the Cantor, all \textbf{sit} at the asterisk.}

\end{rubricbox}

\gresetinitiallines{1}
\gregorioscore{psXXX-antiphon}

\gresetinitiallines{0}
\gregorioscore{psXXX-intonation}

 \begin{latinenglishsection}

\latinenglish{

	\input{../psalsm/psXXX} %%

}{
	\input{../psalms/english/psXXX} %%
}

\end{latinenglishsection}

\begin{rubricbox}

{\color{red}The antiphon is repeated: \textit{\dots}} %%

\end{rubricbox}

\end{latinenglishsection}

%TODO: Verify that the little chapter is fitting for the feast

\section*{Little Chapter (Proverbs 28:20, 27:18)}

\textit{\color{red}The Officiant leads the Little Chapter:}

\begin{latinenglishsection}

\latinenglish{
	Vir fidélis multum laudábitur.~*
	Et qui custos est Dómini sui, glorificábitur.
	\Rbar.~Deo grátias.
}{
	A faithful man shall be much praised, and he that is the keeper of his Lord, shall be glorified.
	 \Rbar.~Thanks be to God.
}

\end{latinenglishsection}

% TODO: Verify that the hymn is correct for the feast (including the responsory after the hymn)

\section*{Hymn}

\textit{\color{red}The Cantor leads the hymn:}

\gresetinitiallines{1}
\gregorioscore{../hymns/te-joseph-celebrent}

{\itshape

   1.  Let Angels chant thy praise, pure spouse of purest Bride,
    While Christendom’s sweet choirs the gladsome strains repeat,
    To tell thy wondrous fame, to raise the pealing hymn,
    Wherewith we all they glory greet.
    
   2.  When doubts and bitter fears thy heavy heart oppressed,
    And filled thy righteous soul with sorrow and dismay,
    An Angel quickly came, the wondrous secret told,
    And drove thy anxious griefs away.
    
    3. Thy arms thy new-born Lord, with tender joy embrace;
    Him then to Egypt’s Land thy watchful care doth bring;
    Him in the Temple’s courts once lost thou dost regain,
    And 'mid thy tears dost greet thy King.
    
   4.  Not till death’s pangs are o’er do others gain their crown,
    But, Joseph, unto thee the blessed lot was given
    While life did yet endure, thy God to see and know,
    As do the Saints above in heaven.
    
    5. Grant us, great Trinity, for Joseph’s holy sake,
    In highest bliss and love, above the stars to reign,
    That we in joy with him may praise our loving God,
    And sing our glad eternal strain. 
}

\textit{\color{red}The Cantor says the following before all reply afterwards:}

\gresetinitiallines{0}
\gabcsnippet{
(c3) <sp>V/</sp>. Gló(h)ri(h)a(h) et(h) di(h)ví(h)ti(h)æ(h) in(h) do(h)mo(h) e(h)jus.(g'_/hvGF'E/fgf.) (::)
}

\gresetinitiallines{0}
\gabcsnippet{
(c3) <sp>R/</sp>. Et(h) jus(h)tí(h)ti(h)a(h) e(h)jus(h) ma(h)net(h) in(h) s<sp>'ae</sp>(h)cu(h)lum(h) s<sp>'ae</sp>(h)cu(h)li.(g'_/hvGF'E/fgf.) (::)
}

\textit{\Vbar.~Glory and riches are in his house.
\Rbar.~And his jutice endureth forever.}

%TODO: Verify that the magnificat antiphon is correct and match the mangificat intonation and pointed text with the tone

\section*{Magnificat}

\textit{\textnormal{Ant Magn.} Behold a faithful and prudent servant, whom the Lord has set over His household.
\textnormal{Cant.} My soul doth magnify the Lord: and my spirit hath rejoiced in God my Saviour.}

\begin{rubricbox}

{\color{red}The Cantor leads by intoning the antiphon and the first verse.}

\end{rubricbox}

\gresetinitiallines{1}
\gregorioscore{magnificat-antiphon-only}

\begin{rubricbox}

{\color{red}All \textbf{stand} and make the sign of the cross with the Cantor.}

\end{rubricbox}

\gresetinitiallines{0}
\gregorioscore{magnificat-intonation}

 \begin{latinenglishsection}

\latinenglish{	
3. Quia respéxit humilitátem ancíllæ \textbf{su}æ:~*
	ecce enim ex hoc beátam me dicent omnes gene\textit{ra}\textit{ti}\textbf{ó}nes.

4. Quia fecit mihi magna qui \textbf{pot}ens est:~* 
	et sanctum \textit{no}\textit{men} \textbf{e}jus.

5. Et misericórdia ejus a progénie in pro\textbf{gé}nies~*
	timén\textit{ti}\textit{bus} \textbf{e}um.

6. Fecit poténtiam in bráchio \textbf{su}o:~*
	dispérsit supérbos mente \textit{cor}\textit{dis} \textbf{su}i.

7. Depósuit poténtes de \textbf{se}de,~*
	et exal\textit{tá}\textit{vit} \textbf{hú}miles.

8. Esuriéntes implévit \textbf{bo}nis:~*
	et dívites dimí\textit{sit} \textit{in}\textbf{á}nes.

9. Suscépit Israël púerum \textbf{su}um,~*
	recordátus misericór\textit{di}\textit{æ} \textbf{su}æ.

10. Sicut locútus est ad patres \textbf{nos}tros,~*
	Abraham et sémini e\textit{jus} \textit{in} \textbf{s\'{\ae}}cula.

}{	
	1. My soul doth magnify the Lord.

2. And my spirit hath rejoiced in God my Saviour.

3. Because he hath regarded the humility of his handmaid; for behold from henceforth all generations shall call me blessed.

4. Because he that is mighty, hath done great things to me; and holy is his name.

5. And his mercy is from generation unto generations, to them that fear him.

6. He hath shewed might in his arm: he hath scattered the proud in the conceit of their heart.

7. He hath put down the mighty from their seat, and hath exalted the humble.

8. He hath filled the hungry with good things; and the rich he hath sent empty away.

9. He hath received Israel his servant, being mindful of his mercy: 

10. As he spoke to our fathers, to Abraham and to his seed for ever. 
}

\end{latinenglishsection}

\textit{\color{red}(bow)}Glória Patri, et \textbf{Fí}lio,~* 
	et Spirí\textit{tu}\textit{i} \textbf{Sanc}to.

\textit{\color{red}(rise)} Sicut erat in princípio, et nunc, et \textbf{sem}per,~*
	et in s\'{\ae}cula sæcu\textit{ló}\textit{rum}. \textbf{A}men.


\begin{rubricbox}

{\color{red}All \textbf{sit} and repeat the antiphon: \textit{\dots}, %%
then \textbf{stand} for the prayer.}

\end{rubricbox}

\vfill\pagebreak

%TODO: Verify (with the antiphonary) that the collect is proper for the season. If it is not in antiphonary, use the missal for the feast.

\section*{Collect}

\textit{\color{red}The Officiant leads the collect:}

\begin{latinenglishsection}

\latinenglish{
	\Vbar.~Dómine exáudi oratiónem meam.\\
	\Rbar.~Et clamor meus ad te véniat.
	
	Orémus.
	Sanctíssimæ Genitrícis tuæ Sponsi, qu\'{\ae}sumus Dómine, méritis adjuvémur:
	ut quod possibílitas nostra non óbtinet, ejus nobis intercessióne donétur.
	Qui vivis.
	\Rbar.~Amen.
}{
	Lord, hear my prayer.
	\Rbar.~And let my cry come unto Thee.
	
	Let us pray.
	We beseech Thee, O Lord, that we may be heped by the merits of the Spouse of Thy most holy Mother,
	so that what we cannot obtain of ourselves, may be given to us through his intercession.
	Who livest.
	\Rbar.~Amen.
}
\end{latinenglishsection}

%TODO: Add commemorations for the date

\textit{\color{red}For commemorations, the Cantor intones the antiphon and says the responsorial prayer afterwards. The Officiant prays the associated collect.}

\textit{\color{red}The Officiant leads the following:}

\latinenglish{
	\Vbar.~Dómine exáudi oratiónem meam.\\
	\Rbar.~Et clamor meus ad te véniat.
}{
	\Vbar.~Lord, hear my prayer. \Rbar.~And let my cry come unto Thee.
	\Vbar.~Let us bless the Lord. \Rbar.~Thanks be to God.
}

\end{latinenglishsection}

\vfill\pagebreak

\textit{\color{red}The Cantor leads the Benedicamus:}

\gresetinitiallines{1}
\gregorioscore{../common/benedicamus-2v-solem}

\textit{\color{red}The Officiant leads the following:}

\begin{latinenglishsection}

\latinenglish{
	\Vbar. Fidélium ánimæ, per misericórdiam Dei, requiéscant in pace. \\
	\Rbar. Amen.
}{
	May the souls of the faithful departed, through the mercy of God, rest in peace. \Rbar.~Amen.
}

\latinenglish{
	Pater noster \textit{(silently)}.
}{
	Our Father\dots
}

\latinenglish{
	\Vbar. Dóminus det nobis suam pacem. \\
	\Rbar. Et vitam ætérnam. Amen.
}{
	May the Lord grant us his peace. \Rbar.~And life eternal. Amen.
}

\end{latinenglishsection}

%TODO: Add the Marian Anthem for the season and verify that the oration afterwards is correct

\section*{Marian Anthem}

\textit{\color{red}The Cantor leads the Marian anthem and responses afterwards; the Officiant leads the ending collect:}

\gresetinitiallines{1}
\gregorioscore{../marian-anthems/ave-regina-caelorum-simple}

{\itshape
	Hail, O Queen of Heaven enthroned.
	Hail, by angels mistress owned.
	Root of Jesse, Gate of Morn
	whence the world's true light was born.
	
	Glorious Virgin, Joy to thee,
	loveliest whom in heaven they see;
	fairest thou, where all are fair,
	plead with Christ our souls to spare.
}

\begin{latinenglishsection}

\latinenglish{
	\Vbar. Dignáre me laudáre te, Virgo sacráta.\\
	\Rbar. Da mihi virtútem contra hostes tuos.
	
	Orémus.
	Concéde, miséricors Deus, fragilitáti nostræ præsídium:
	ut qui sanctæ Dei Genitrícis memóriam ágimus;
	intercessiónis ejus auxílio, a nostris iniquitátibus resurgámus.
	Per eúmdem Christum Dóminum nostrum.
	\Rbar.~Amen.
}{
	Vouchsafe that I may praise thee, O sacred Virgin.
	\Rbar.~Give me stength against thine enemies.
	
	Let us pray.
	Grant, O merciful God, to our weak natures Thy protection,
	that we who commemorate the holy Mother of God may,
	by the help of her intercession, arise from our iniquities.
	Through the same Christ our Lord.
	\Rbar.~Amen.
}
\end{latinenglishsection}

\textit{\color{red}The Officiant says the following:}

\begin{latinenglishsection}

\latinenglish{
	\Vbar. Divínum auxílium máneat semper nobíscum.\\
	\Rbar. Amen.
}{
	May the divine assistance remain always with us.
	\Rbar.~Amen.
}
\end{latinenglishsection}

\begin{rubricbox}

After the Office, all \textbf{kneel} and pray in silence for a time.

\end{rubricbox}

\end{document}