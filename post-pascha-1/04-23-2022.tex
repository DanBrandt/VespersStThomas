\documentclass{../vespers-sheet}

\begin{document}

\chapter*{Vespers Insert for Saturday, April 23}

 \textit{Vespers today is based on the following Sunday, which is the first after Easter, also called Low Sunday.
 All is said as in the Eastertide booklet until the Little Chapter, except that the antiphon is sung in full before the psalms, as follows:}
 
\gresetinitiallines{1}
\gregorioscore{pss-intonation}

\section*{Little Chapter (1 John 5:4)}

\begin{latinenglishsection}

 \latinenglish{
	 Caríssimi: Omne, quod natum est ex Deo, vincit mundum:~\GreDagger\
	 et hæc est victoria, quæ vincit mundum,~*
	 fides nostra.
 	\Rbar.~Deo gratias.
 }{
 	Beloved: Whatsoever is born of God, overcometh the world:
 	and this is the victory which overcometh the world, our faith. 
 	\Rbar.~Thanks be to God.
 }
 
\end{latinenglishsection}

\section*{Hymn}

\begin{rubricbox}

The hymn \textit{Ad cenam agni providi} and versicle \textit{\Rbar.~Mane nobiscum\dots} are said as given in the booklet for Eastertide.

\end{rubricbox}

\section*{Magnificat}

\textit{\textnormal{Ant. Magn.} When it was late that same day, the first of the week,
though the doors where the disciples gathered had been closed, Jesus stood in the midst and said to them: Peace be to you, alleluia.}

\begin{rubricbox}

After the leader intones up to the asterisk, all \textbf{sit} and join in singing.

\end{rubricbox}

\gresetinitiallines{1}
\gregorioscore{magnificat-antiphon-only}

\begin{rubricbox}

All \textbf{stand} and make the sign of the cross with the cantor.

\end{rubricbox}

\gresetinitiallines{0}
\gregorioscore{magnificat-intonation}

3. \textit{Quia} respéxit humilitátem an\textbf{cíl}læ \textbf{su}æ:~*
	ecce enim ex hoc beátam me dicent omnes gene\textit{ra}\textit{ti}\textbf{ó}nes.

4. \textit{Quia} fecit mihi \textbf{ma}gna qui \textbf{pot}ens est:~*
	et sanctum \textit{no}\textit{men} \textbf{e}jus.

5. \textit{Et\ miseri}córdia ejus a progénie \textbf{in} pro\textbf{gé}nies~*
	timén\textit{ti}\textit{bus} \textbf{e}um.

6. \textit{Fecit} poténtiam in \textbf{brá}chio \textbf{su}o:~*
	dispérsit supérbos mente \textit{cor}\textit{dis} \textbf{su}i.

7. \textit{Depósuit} pot\textbf{én}tes de \textbf{se}de,~*
	et exal\textit{tá}\textit{vit} \textbf{hú}miles.

8. \textit{Esuri}éntes im\textbf{plé}vit \textbf{bo}nis:~*
	et dívites dimí\textit{sit} \textit{in}\textbf{á}nes.

9. \textit{Suscépit} Israël \textbf{pú}erum \textbf{su}um,~*
	recordátus misericór\textit{di}\textit{æ} \textbf{su}æ.

10. \textit{Sicut} locútus est ad \textbf{pa}tres \textbf{nos}tros,~*
	Abraham et sémini e\textit{jus} \textit{in} \textbf{s\'{\ae}}cula.

\textit{(bow)} \textit{Glóri}a \textbf{Pa}tri, et \textbf{Fí}lio,~*
	et Spirí\textit{tu}\textit{i} \textbf{Sanc}to.

\textit{(rise)} \textit{Sicut} erat in princípio, et \textbf{nunc}, et \textbf{sem}per,~*
	et in s\'{\ae}cula sæcu\textit{ló}\textit{rum}. \textbf{A}men.

\begin{rubricbox}

All \textbf{sit} and repeat the antiphon: \textit{Cum esset sero\dots},
then \textbf{stand} for the prayer.

\end{rubricbox}

\section*{Collect}

\begin{latinenglishsection}

\latinenglish{
	\Vbar.~Dómine exáudi oratiónem meam.\\
	\Rbar.~Et clamor meus ad te véniat.
	
	Orémus.
	Præsta, qu\'{\ae}sumus, omnípotens Deus:
	ut, qui paschália festa perégimus,
	hæc, te largiénte, móribus et vita teneámus.
	Per Dóminum.
	\Rbar.~Amen.
}{
	Lord, hear my prayer.
	\Rbar.~And let my cry come unto Thee.
	
	Let us pray.
	Grant, we beseech Thee, almighty God, that we who have celebrated the Paschal Feast,
	may, by Thy bounty, retain its fruits in our daily habits and behavior.
	Through our Lord.
	\Rbar.~Amen.
}

\end{latinenglishsection}

\begin{latinenglishsection}

\latinenglish{
	\Rbar.~Amen. \\
	\Vbar.~Dómine exáudi oratiónem meam.\\
	\Rbar.~Et clamor meus ad te véniat.
}{
	\Rbar.~Amen. \Vbar.~Lord, hear my prayer. \Rbar.~And let my cry come unto Thee.
	\Vbar.~Let us bless the Lord. \Rbar.~Thanks be to God.
}

\end{latinenglishsection}

\gresetinitiallines{1}
\gregorioscore{../common/benedicamus-tp}

\begin{latinenglishsection}

\latinenglish{
	\Vbar. Fidélium ánimæ, per misericórdiam Dei, requiéscant in pace. \\
	\Rbar. Amen.
}{
	May the souls of the faithful departed, through the mercy of God, rest in peace. \Rbar.~Amen.
}

\latinenglish{
	Pater noster \textit{(silently)}.
}{
	Our Father\dots
}

\latinenglish{
	\Vbar. Dóminus det nobis suam pacem. \\
	\Rbar. Et vitam ætérnam. Amen.
}{
	May the Lord grant us his peace. \Rbar.~And life eternal. Amen.
}

\end{latinenglishsection}

\begin{rubricbox}

The Marian anthem \textit{Regina Caeli} follows (no. 221 in the \textit{Traditional Roman Hymnal}).

\end{rubricbox}

%% doesn't fit on the page
%\begin{latinenglishsection}
%
%\latinenglish{
%	\Vbar. Divínum auxílium máneat semper nobíscum.\\
%	\Rbar. Amen.
%}{
%	May the divine assistance remain with us always. \Rbar.~Amen.
%}
%
%\end{latinenglishsection}

\end{document}