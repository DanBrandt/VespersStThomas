\documentclass[12pt]{../../vespers-sheet} %booklet
\usepackage{multicol}

\begin{document}

% TODO: Update the title for the specific feast
\chapter*{First Vespers of Our Lady of the Rosary}

\begin{center}
\includegraphics[width=\textwidth]{lady-of-the-rosary}
\end{center}

\vfill\pagebreak

%\section*{Beginning of the Office}

\begin{rubricbox}

{\color{red} All make the sign of the cross with the Officiant as he intones:}

\end{rubricbox}

% TODO: Make sure that the tone of the deus adjutorium matches the season primarily and the solemnity of the feast secondarily
 \gresetinitiallines{1}
\gregorioscore{../../common/deus-in-adjutorium}

\textit{
O God, come to my assistance.
{\color{red}\Vbar.}~O Lord, make haste to help me.
Glory be to the Father, and to the Son, and to the Holy Spirit,
as it was in the beginning, is now, and ever shall be, world without end. Amen.
Praise to Thee, O Lord, King of endless glory.}

%\vfill\pagebreak}

\section*{Psalm 109}

\textit{\textnormal{Ant. 1.} Who is this, * fair as a dove, like a rose-tree planted beside the rivers of waters?
 \textnormal{Ps.} The Lord said to my lord, sit thou at My right hand.}

\gresetinitiallines{1}
\gregorioscore{ps109-antiphon}

\begin{rubricbox}

{\color{red} The incipit of each Psalm is only used on the first verse.}

\end{rubricbox}

\gresetinitiallines{0}
\gregorioscore{ps109-intonation}

 \begin{latinenglishsection}

\latinenglish{

	\input{../../psalms/ps109-1} %% 

}{
	\input{../../psalms/english/ps109} %%
}

\end{latinenglishsection}

\gresetinitiallines{1}
\gregorioscore{ps109-antiphon}

%%

\section*{Psalm 112}

\textit{\textnormal{Ant. 2.} A virgin most mighty, * like the tower of David, whereon there hung a thousand bucklers, all the shields of valiant men.
 \textnormal{Ps.} Praise the Lord, ye children: praise ye the name of the Lord.}

\gresetinitiallines{1}
\gregorioscore{ps112-antiphon}

\gresetinitiallines{0}
\gregorioscore{ps112-intonation}

 \begin{latinenglishsection}

\latinenglish{

	\input{../../psalms/ps112-6} %%

}{
	\input{../../psalms/english/ps112} %%
}

\end{latinenglishsection}

\gresetinitiallines{1}
\gregorioscore{ps112-antiphon}

%%

\section*{Psalm 121}

\textit{\textnormal{Ant. 3.} Hail Mary, full of grace, the Lord is with thee, blessed art thou among women.
 \textnormal{Ps.} I rejoiced at the things that: were said to me: We shall go into the house of the Lord.}

\gresetinitiallines{1}
\gregorioscore{ps121-antiphon}

\gresetinitiallines{0}
\gregorioscore{ps121-intonation}

 \begin{latinenglishsection}

\latinenglish{

	\input{../../psalms/ps121-1} %%

}{
	\input{../../psalms/english/ps121} %%
}

\end{latinenglishsection}

\gresetinitiallines{1}
\gregorioscore{ps121-antiphon}

%%

\section*{Psalm 126}

\textit{\textnormal{Ant. 4.} The Lord hath blessed thee * by His power, because through thee He hath brought our enemies to nought.
 \textnormal{Ps.} Unless the Lord build the house, they labour in vain that build it.}

\gresetinitiallines{1}
\gregorioscore{ps126-antiphon}

\gresetinitiallines{0}
\gregorioscore{ps126-intonation}

 \begin{latinenglishsection}

\latinenglish{

	\input{../../psalms/ps126-7a} %%

}{
	%1. Unless the Lord build the house, they labour in vain that build it.

2. Unless the Lord keep the city, he watcheth in vain that keepeth it.

3. It is vain for you to rise before light, rise ye after you have rested, you that eat the bread of sorrow.

4. When he shall give sleep to his beloved: behold the inheritance of the Lord are children: the reward, the fruit of the womb.

5. As arrows in the hand of the mighty, so the children of them that have been cast out.

6. Blessed is the man that hath filled his desire with them;
he shall not be confounded when he shall speak to his enemies in the gate. 

Glory be. %%
}

\end{latinenglishsection}

\gresetinitiallines{1}
\gregorioscore{ps126-antiphon}

%%

\section*{Psalm 147}

\textit{\textnormal{Ant. 5.} The daughters of Sion saw her in her spring time, amidst the flowers of the roses, and called her most blessed.
 \textnormal{Ps.} Praise the Lord, O Jerusalem: praise thy God, O Sion.}

\gresetinitiallines{1}
\gregorioscore{ps147-antiphon}

\gresetinitiallines{0}
\gregorioscore{ps147-intonation}

 \begin{latinenglishsection}

\latinenglish{

	\input{../../psalms/ps147-3a} %%

}{
	\input{../../psalms/english/ps147} %%
}

\end{latinenglishsection}

\gresetinitiallines{1}
\gregorioscore{ps147-antiphon}

%%

\vfill\pagebreak

\section*{Little Chapter (Sirach 24:25; 39:17)}

\textit{\color{red}The Officiant leads the Little Chapter:}

\begin{latinenglishsection}

\latinenglish{
	In me grátia omnis viæ et veri\textbf{tá}\textit{tis}, {\color{red}\GreDagger}\ in me omnis spes vi\textbf{tæ} \textit{et vir}\textbf{tú}tis. Ego, quasi rosa plantáta super rivos aquárum, fructificávi.
	
	{\color{red}\Rbar.}~Deo grátias.
}{
	In me is all grace of the way and of the truth; in me is all hope of life and virtue; I have flowered forth like a rose planted by the brooks of water.
	 {\color{red}\Rbar.}~Thanks be to God.
}

\end{latinenglishsection}

\section*{Hymn}

\textit{\color{red}The Cantor leads the hymn:}

\gresetinitiallines{1}
\gregorioscore{../../hymns/te-gestientem-gaudiis}

{\itshape
	1.  The gladness of thy motherhood,
	The anguish of thy suffering,
	The glory now that crowns thy brow,
	O Virgin-Mother, we would sing.
	
	2. Hail, blessed Mother, full of joy
	In thy consent, thy visit too;
	Joy in the birth of Christ on earth,
	Joy in him lost and found anew.
	
	3. Hail, sorrowing in his agony—
	The blows, the thorns that pierced his brow;
	The heavy wood, the shameful rood—
	Yea! Queen and chief of martyrs thou.
	
	4. Hail, in the triumph of thy Son,
	The quickening flames of Pentecost;
	Shining a Queen in light serene,
	When all the world is tempest-tost.
	
	5. O come, ye nations, roses bring,
	Culled from these mysteries divine,
	And for the Mother of your King
	With loving hands your chaplets twine.
	
	6. We lay our homage at thy feet,
	Lord Jesus, thou the Virgin's Son,
	With Father and with Paraclete,
	Reigning while endless ages run.
	Amen.
}

\textit{\color{red}The Cantor says the following before all reply afterwards:}

\gresetinitiallines{0}
\gabcsnippet{
(c3) <c><sp>V/</sp>.</c> Re(h)gí(h)na(h) sac(h)ra(h)tís(h)si(h)mi(h) Ro(h)sá(h)rii(h), o(h)ra(h) pro(h) no(h)bis.(g'_/hvGF'E/fgf.) (::)
}

\gresetinitiallines{0}
\gabcsnippet{
(c3) <c><sp>R/</sp>.</c> Ut(h) dig(h)ni(h) ef(h)fi(h)ci(h)á(h)mur(h) pro(h)mis(h)si(h)ó(h)ni(h)bus(h) Chris(h)ti.(g'_/hvGF'E/fgf.) (::)
}

\textit{{\color{red}\Vbar.}~Queen of the most holy Rosary, pray for us.
{\color{red}\Rbar.}~That we may be made worthy of the promises of Christ.}

\vfill\pagebreak

\section*{Magnificat}

\textit{\textnormal{Ant Magn.} Blessed Mother * and Inviolate Maiden, Glorious Queen of the World, may all that keep the solemn Feast of thy Most Holy Rosary feel the might of thine assistance.
\textnormal{Cant.} My soul doth magnify the Lord: and my spirit hath rejoiced in God my Saviour.}

\begin{rubricbox}

{\color{red}The Cantor leads by intoning the antiphon and the first verse.}

\end{rubricbox}

\gresetinitiallines{1}
\gregorioscore{magnificat-antiphon-only}

\begin{rubricbox}

{\color{red}All \textbf{remain standing} and make the sign of the cross with the Cantor. The incipit of the Magnificat is used at the beginning of every verse:}

\end{rubricbox}

\gresetinitiallines{0}
\gregorioscore{magnificat-intonation}

 \begin{latinenglishsection}

\latinenglish{	
	3. Quia respéxit humilitátem ancíllæ \textbf{su}æ:~* ecce enim ex hoc beátam me dicent omnes gene\textit{ra}\textit{ti}\textbf{ó}nes.

	4. Quia fecit mihi magna qui \textbf{pot}ens est:~* et sanctum \textit{no}\textit{men} \textbf{e}jus.
	
	5. Et misericórdia ejus a progénie in pro\textbf{gé}nies~* timén\textit{ti}\textit{bus} \textbf{e}um.
	
	6. Fecit poténtiam in bráchio \textbf{su}o:~* dispérsit supérbos mente \textit{cor}\textit{dis} \textbf{su}i.
	
	7. Depósuit poténtes de \textbf{se}de,~* et exal\textit{tá}\textit{vit} \textbf{hú}miles.
	
	8. Esuriéntes implévit \textbf{bo}nis:~* et dívites dimí\textit{sit} \textit{in}\textbf{á}nes.
	
	9. Suscépit Israël púerum \textbf{su}um,~* recordátus misericór\textit{di}\textit{æ} \textbf{su}æ.
	
	10. Sicut locútus est ad patres \textbf{nos}tros,~* Abraham et sémini e\textit{jus} \textit{in} \textbf{s\'{\ae}}cula.

}{	
	\input{../../psalms/english/magnificat}
}

\end{latinenglishsection}

\textit{\color{red}(bow)} Glória Patri, et \textbf{Fí}lio,~* et Spirí\textit{tu}\textit{i} \textbf{Sanc}to.

\textit{\color{red}(rise)} Sicut erat in princípio, et nunc, et \textbf{sem}per,~* et in s\'{\ae}cula sæcu\textit{ló}\textit{rum}. \textbf{A}men.

\gresetinitiallines{1}
\gregorioscore{magnificat-antiphon-only}

\vfill\pagebreak

%TODO: Verify (with the antiphonary) that the collect is proper for the season. If it is not in antiphonary, use the missal for the feast.

\section*{Collect}

\textit{\color{red}The Officiant leads the collect:}

\begin{latinenglishsection}

\latinenglish{
	{\color{red}\Vbar.}~Dómine exaudi orationem meam.\\
	{\color{red}\Rbar.}~Et cum spíritu túo.
	
	Orémus.
	Deus, cujus Unigénitus per vitam, mortem et resurrectiónem suam nobis salútis ætérnæ pr\'{\ae}mia compa\textbf{rá}\textbf{vit}: {\color{red}\GreDagger}\ concéde, qu\'{\ae}sumus; ut hæc mystéria sacratíssimo beátæ Maríæ Vírginis Rosári\textbf{o} \textit{reco}\textbf{lén}tes, et imitémur quod cóntinent, et quod promíttunt, assequámur.
}{
	{\color{red}\Vbar.} Lord, hear my prayer.
	{\color{red}\Rbar.}~And let my cry come unto Thee.
	
	Let us pray.
	Grant, we beseech thee, O Lord God, unto all thy servants, that they may remain continually in the enjoyment of soundness both of mind and body, and by the glorious intercession of the Blessed Mary, always a Virgin, may be delivered from present sadness, and enter into the joy of thine eternal gladness.
}
\end{latinenglishsection}

\textit{\color{red}For commemorations, the Cantor intones the antiphon and says the responsorial prayer afterwards. The Officiant prays the associated collect.}

\section*{Commemoration of the 19th Sunday After Pentecost (October 8)}

\textit{\textnormal{Ant.} The sun shone upon the shields of gold, and the mountains glistened therewith, and the army of the heathens was spread abroad.}

\gresetinitiallines{1}
\gregorioscore{commemoration-antiphon-1}

\begin{latinenglishsection}

\latinenglish{
	{\color{red}\Vbar.}.~Vespertína orátio ascéndat ad te, \textbf{Dó}mine.\\
	{\color{red}\Rbar.}.~Et descéndat super nos misericórdia \textbf{tu}a.
	
	Orémus.
	
	Omnípotens et miséricors Deus, univérsa nobis adversántia propitiátus ex\textbf{clú}\textit{de}: {\color{red}\GreDagger}\ ut mente et córpore páriter expedíti, quæ tua sunt, líberis méntibus exsequámur.
}{
	{\color{red}\Vbar.}.~Let the evening prayer ascend unto thee, O Lord.
	{\color{red}\Rbar.}.~And let there descend upon us thy mercy.
	
	Let us pray.
	O Almighty and most merciful God, of thy bountiful goodness keep us, we beseech thee, from all things that may hurt us; that we, being ready both in body and soul, may cheerfully accomplish those things that Thou wouldest have done.
}

\end{latinenglishsection}

\vfill\pagebreak

\section*{Commemoration of St. Brigid of Sweden, Widow (October 8)}

\textit{\textnormal{Ant.} The kingdom of heaven is like unto a merchantman seeking goodly pearls, who, when he had found one pearl of great price, went and sold all that he had, and bought it.}

\gresetinitiallines{1}
\gregorioscore{commemoration-antiphon-2}

\begin{latinenglishsection}

\latinenglish{
	{\color{red}\Vbar.}.~Spécie tua et pulchritúdine \textbf{tu}a.\\
	{\color{red}\Rbar.}.~Inténde, próspere procéde, et \textbf{reg}na.
	
	Orémus.
	
	Dómine, Deus noster, qui beátæ Birgíttæ per Fílium tuum unigénitum secréta cæléstia reve\textbf{lás}\textit{ti}: {\color{red}\GreDagger}\ ipsíus pia intercessióne da nobis \textbf{fá}\textit{mulis} \textbf{tu}is; in revelatióne sempitérnæ glóriæ tuæ gaudére lætántes. {\color{red}*} 
	Per eúmdem Dóminum nostrum Jesum Christum Fílium tuum, qui tecum vivit et regnat in unitáte Spíritus Sancti, Deus, per ómnia s\'{\ae}cula sæculórum.
	
	{\color{red}\Rbar.}~Amen.
}{
	{\color{red}\Vbar.}.~Let the evening prayer ascend unto thee, O Lord.
	{\color{red}\Rbar.}.~And let there descend upon us thy mercy.
	
	Let us pray.
	O Almighty and most merciful God, of thy bountiful goodness keep us, we beseech thee, from all things that may hurt us; that we, being ready both in body and soul, may cheerfully accomplish those things that Thou wouldest have done.
	Through the same Jesus Christ, thy Son, Our Lord, Who liveth and reigneth with thee in the unity of the Holy Ghost, God, world without end.
	
	{\color{red}\Rbar.}~Amen.
}

\end{latinenglishsection}

\textit{\color{red}The Officiant leads the following:}

\begin{latinenglishsection}

\latinenglish{
	{\color{red}\Vbar.}~Dómine exaudi orationem meam.\\
	{\color{red}\Rbar.}~Et cum spíritu túo.
}{
	{\color{red}\Vbar.}~O Lord, hear my prayer. {\color{red}\Rbar.}~And let my cry come unto Thee.
	{\color{red}\Vbar.}~Let us bless the Lord. {\color{red}\Rbar.}~Thanks be to God.
}

\end{latinenglishsection}

%\vfill\pagebreak

\textit{\color{red}The Cantor leads the Benedicamus:}

\gresetinitiallines{0}
\gregorioscore{../../common/benedicamus-bvm}

\textit{\color{red}The Officiant leads the following:}

\begin{latinenglishsection}

\latinenglish{
	{\color{red}\Vbar.} Fidélium ánimæ, per misericórdiam\\ Dei, requiéscant in pace. \\
	{\color{red}\Rbar.} Amen.
}{
	May the souls of the faithful departed, through the mercy of God, rest in peace. {\color{red}\Rbar.}~Amen.
}

\end{latinenglishsection}

\section*{Marian Anthem}

\textit{\color{red}The Marian anthem is said kneeling, since it is neither a Sunday nor a first-class feast. It and the responses afterwards are led by the Cantor; the Officiant leads the ending collect:}

\gresetinitiallines{1}
\gregorioscore{../../marian-anthems/salve-regina-simple}

{\itshape
	Hail, holy Queen, Mother of mercy, our life, our sweetness and our hope. 
	To thee do we cry, poor banished children of Eve. 
	To thee do we send up our sighs, mourning and weeping in this valley of tears. 
	Turn, then, most gracious advocate, thine eyes of mercy toward us, 
	and after this, our exile, show unto us the blessed fruit of thy womb, Jesus. 
	O clement, O loving, O sweet Virgin Mary.
}

\begin{latinenglishsection}

\latinenglish{
	{\color{red}\Vbar.} Ora pro nóbis sáncta Déi \textbf{Gé}nitrix.\\
	{\color{red}\Rbar.} Ut dígni efficiámur promissiónibus \textbf{Chrí}sti.
	
	Orémus.
	Concéde, miséricors Deus, fragilitáti nostræ præsídium:{\color{red}\GreDagger}
	ut qui sanctæ Dei\\ Genitrícis me\textbf{mó}\textit{riam} \textbf{á}gimus;
	intercessiónis ejus auxílio, a nostris iniquitátibus resurgámus. {\color{red}*}
	Per eúmdem Christum Dóminum nostrum.
	{\color{red}\Rbar.}~Amen.
}{
	{\color{red}\Vbar.}~Pray for us, O holy Mother of God.
	{\color{red}\Rbar.}~That we may be made worthy of the promises of Christ.
	
	Let us pray.
	Almighty and everlasting God, Who by the working of the Holy Spirit didst prepare both body and soul of the glorious Virgin Mother, Mary, that she might deserve to be made a worthy dwelling for Thy Son, grant that we who rejoice in her memory, may, by her loving intercession, be delivered from present evils and from lasting death, through the same Christ our Lord.
	{\color{red}\Rbar.}~Amen.
}
\end{latinenglishsection}

\textit{\color{red}The Officiant says the following in a low recto tono:}

\begin{latinenglishsection}

\latinenglish{
	{\color{red}\Vbar.} Divínum auxílium máneat semper nobíscum.\\
	{\color{red}\Rbar.} Amen.
}{
	{\color{red}\Vbar.}~May the divine assistance remain always with us.
	{\color{red}\Rbar.}~Amen.
}

\end{latinenglishsection}

\begin{rubricbox}

{\color{red}After the Office, all \textbf{kneel} and pray in silence for a time.}

\end{rubricbox}

\end{document}