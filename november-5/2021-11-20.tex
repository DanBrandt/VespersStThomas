\documentclass{../vespers-sheet}

\begin{document}

\chapter*{Vespers Insert for Saturday, Nov. 20, 2021.}

 \textit{Vespers today is based on the following Sunday, which is the Fifth of November and the 26th and last after Pentecost. All is said as in the main booklet until the hymn, which is modified because we are commemorating a feast of Our Lady (her presentation in the temple).}
 
 \section*{Hymn}

% including the entire hymn/translation and V/ and R/ here because it doesn't change the page length of the insert anyways
\gresetinitiallines{1}
\gregorioscore{../hymns/o-lux-beata-trinitas-bvm-dox}

{\itshape
	1. O Trinity of blessed Light,
	O Unity of sovereign might,
	as now the fiery sun departs,
	shed Thou Thy beams within our hearts.
	
	2. To Thee our morning song of praise,
	to Thee our evening prayer we raise;
	Thee may our glory evermore
	in lowly reverence adore.
	
	3. All honor, laud, and glory be,
	O Jesu Virgin-born, to Thee,
	Whom with the Father we adore,
	and Holy Ghost for evermore. Amen. 
}

%\vfill
%\pagebreak

\gresetinitiallines{0}
\gabcsnippet{
(c3) <sp>V/</sp>. Ves(h)per(h)tí(h)na(h) o(h)rá(h)ti(h)o(h) as(h)cén(h)dat(h) ad(h) te(h) Dó(h)mi(h)ne.(g'_/hvGF'E/fgf.) (::)
}

\gresetinitiallines{0}
\gabcsnippet{
(c3) <sp>R/</sp>. Et(h) des(h)cén(h)dat(h) su(h)per(h) nos(h) mi(h)se(h)ri(h)cór(h)di(h)a(h) tu(h)a.(g'_/hvGF'E/fgf.) (::)
}

\section*{Magnificat}

\gresetinitiallines{1}
\gregorioscore{magnificat-intonation}

3. \textit{Quia} respéxit humilitátem an\textbf{cíl}læ \textbf{su}æ:~*
	ecce enim ex hoc beátam me dicent omnes gene\textit{ra}\textit{ti}\textbf{ó}nes.

4. \textit{Quia} fecit mihi \textbf{ma}gna qui \textbf{pot}ens est:~*
	et sanctum \textit{no}\textit{men} \textbf{e}jus.

5. \textit{Et miseri}córdia ejus a progénie \textbf{in} pro\textbf{gé}nies~*
	timén\textit{ti}\textit{bus} \textbf{e}um.

6. \textit{Fecit} poténtiam in \textbf{brá}chio \textbf{su}o:~*
	dispérsit supérbos mente \textit{cor}\textit{dis} \textbf{su}i.

7. \textit{Depósu}it pot\textbf{én}tes de \textbf{se}de,~*
	et exal\textit{tá}\textit{vit} \textbf{húmi}les.

8. \textit{Esuri}éntes im\textbf{plé}vit \textbf{bo}nis:~*
	et dívites dimí\textit{sit} \textit{in}\textbf{á}nes.

9. \textit{Suscépit} Israël \textbf{pú}erum \textbf{su}um,~*
	recordátus misericór\textit{di}\textit{æ} \textbf{su}æ.

10. \textit{Sicut} locútus est ad \textbf{pa}tres \textbf{nos}tros,~*
	Abraham et sémini e\textit{jus} \textit{in} \textbf{s\'{\ae}cu}la.

{\color{red}\textit{(bow)}} \textit{Glóri}a \textbf{Pa}tri, et \textbf{Fí}lio,~*
	et Spirí\textit{tu}\textit{i} \textbf{Sanc}to.

{\color{red}\textit{(rise)}} \textit{Sicut} erat in princípio, et \textbf{nunc}, et \textbf{sem}per,~*
	et in s\'{\ae}cula sæcu\textit{ló}\textit{rum}. \textbf{A}men.

\begin{rubricbox}

All \textbf{sit} and sing the antiphon together.

\end{rubricbox}

\gresetinitiallines{0}
\gregorioscore{magnificat-antiphon}

\textit{\textnormal{Ant.} Upon thy walls, O Jerusalem, I have set watchmen; day and night they shall not cease from praising the Name of the Lord.}

\begin{rubricbox}

All \textbf{stand} for the reading of the collect prayer by the leader.

\end{rubricbox}

\section*{Collect}

\begin{latinenglishsection}

\latinenglish{
	\Vbar.~Dómine exáudi oratiónem meam.\\
	\Rbar.~Et clamor meus ad te véniat.
	
	Orémus.
	Excita, qu\'{\ae}sumus Dómine, tuórum fidélium voluntátes: ut divíni óperis fructum propénsius exsequéntes,
	pietátis tuæ remédia majóra percípiant. Per Dóminum.
	\Rbar.~Amen.
}{

	Lord, hear my prayer.
	\Rbar.~And let my cry come unto Thee.
	
	Let us pray.
	Stir up the wills of Thy faithful people, we beseech Thee, O Lord; that they more earnestly seeking the fruit of good works,
	may receive more abundantly the gifts of Thy loving kindness. Through our Lord.
	\Rbar.~Amen.
}

\end{latinenglishsection}

\section*{Commemoration of St. Felix of Valois, Confessor (Nov. 20)}

\textit{\textnormal{Ant.} This man, despising the world, and triumphing over earthly things, hath laid up treasures in heaven by word and deed.}

\begin{rubricbox}

After the cantor intones up to the asterisk, all \textbf{sit} and join in singing.

\end{rubricbox}

\gresetinitiallines{1}
\gregorioscore{../commemorations/conf-non-pont-2v}

\begin{rubricbox}

All \textbf{stand} for the prayer.

\end{rubricbox}

\begin{latinenglishsection}

\latinenglish{
	\Vbar.~Justum dedúxit Dóminus per vias rectas.\\
	\Rbar.~Et osténdit illi regnum Dei.
	
	Orémus.
	Deus, qui beátum Felícem Confessórem tuum ex erémo ad munus rediméndi captívos c\'{\ae}litus vocáre dignátus es:
	præsta, qu\'{ae}sumus; ut per grátiam tuam ex peccatórum nostrórum captivitáte, ejus intercessióne liberáti,
	ad cæléstem pátriam perducámur.
}{
	The Lord led the just in right paths.
	\Rbar.~And showed him the kingdom of God.
	
	Let us pray.
	O God, Who didst vouchsafe by a voice from heaven to call blessed Felix to the work of the ransoming of captives:
	grant, we beseech Thee, that his holy prayers may free us from the bondage of sin, and may safely lead us to our heavenly fatherland.
}

\end{latinenglishsection}

\section*{Commemoration of the Presention of Our Lady in the Temple (Nov. 21)}

\textit{\textnormal{Ant.} O blessed Mother of God, Mary, ever Virgin, temple of the Lord, sanctuary of the Holy Spirit, thou alone without a rival, hast pleased our Lord Jesus Christ, alleluia.}

\begin{rubricbox}

After the cantor intones up to the asterisk, all \textbf{sit} and join in singing.

\end{rubricbox}

\gresetinitiallines{1}
\gregorioscore{comm-presentation}

\begin{rubricbox}

All \textbf{stand} for the prayer (on the next page).

\end{rubricbox}

\vfill
\pagebreak

\begin{latinenglishsection}

\latinenglish{
	\Vbar.~Dignáre me laudáre te Virgo sacráta.\\
	\Rbar.~Da mihi virtútem contra hostes tuos.
	
	Orémus.
	Deus, qui beátam Maríam semper Vírginem, Spíritus Sancti habitáculum, hodiérna die in templo præsentári voluísti:
	præsta, qu\'{\ae}sumus, ut ejus intercessióne, in templo glóriæ tuæ præsentári mereámur.
	Per Dóminum\dots in unitáte ejúsdem.
}{
	Grant that I may praise thee, O holy virgin.
	\Rbar.~Give me strength against thy enemies.
	
	Let us pray.
	O God, who wast pleased that on this day the blessed Mary ever-Virgin, the dwelling-place of the Holy Ghost,
	should be presented in the temple; grant we beseech Thee,
	that through her prayers we may be found worthy to be presented in the temple of Thy glory.
	Through our Lord\dots in the unity of the same.
}

\end{latinenglishsection}

\begin{latinenglishsection}

\latinenglish{
	\Rbar.~Amen. \\
	\Vbar.~Dómine exáudi oratiónem meam.\\
	\Rbar.~Et clamor meus ad te véniat.
}{
	\Rbar.~Amen. \Vbar.~Lord, hear my prayer. \Rbar.~And let my cry come unto Thee.
	\Vbar.~Let us bless the Lord. \Rbar.~Thanks be to God.
}

\end{latinenglishsection}

\gresetinitiallines{1}
\gregorioscore{../common/benedicamus-sunday}

\begin{latinenglishsection}

\latinenglish{
	\Vbar. Fidélium ánimæ, per misericórdiam Dei, requiéscant in pace. \\
	\Rbar. Amen.
}{
	May the souls of the faithful departed, through the mercy of God, rest in peace. \Rbar.~Amen.
}

\latinenglish{
	Pater noster \textit{(silently)}.
}{
	Our Father\dots
}

\latinenglish{
	\Vbar. Dóminus det nobis suam pacem. \\
	\Rbar. Et vitam ætérnam. Amen.
}{
	May the Lord grant us his peace. \Rbar.~And life eternal. Amen.
}

\end{latinenglishsection}

\begin{rubricbox}

The Marian anthem \textit{Salve Regina} follows (no. 224 in the \textit{Traditional Roman Hymnal}).

\end{rubricbox}

\end{document}